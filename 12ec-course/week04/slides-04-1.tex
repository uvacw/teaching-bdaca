\documentclass[compress]{beamer}
% !TeX document-id = {f19fb972-db1f-447e-9d78-531139c30778}
% !BIB program = biber

%\documentclass[handout]{beamer}
%\documentclass[compress]{beamer}
\usepackage[T1]{fontenc}
\usetheme[block=fill,subsectionpage=progressbar,sectionpage=progressbar]{metropolis} 
\usepackage{graphicx}

\usepackage{wasysym}
\usepackage{etoolbox}
\usepackage[utf8]{inputenc}

\usepackage{pifont}

\usepackage{threeparttable}
\usepackage{subcaption}

\usepackage{tikz-qtree}
\usepackage{neuralnetwork}

\setbeamercovered{still covered={\opaqueness<1->{5}},again covered={\opaqueness<1->{100}}}


\usepackage{listings}

\lstset{
	basicstyle=\scriptsize\ttfamily,
	columns=flexible,
	breaklines=true,
	numbers=left,
	%stepsize=1,
	numberstyle=\tiny,
	backgroundcolor=\color[rgb]{0.85,0.90,1}
}



\lstnewenvironment{lstlistingoutput}{\lstset{basicstyle=\footnotesize\ttfamily,
		columns=flexible,
		breaklines=true,
		numbers=left,
		%stepsize=1,
		numberstyle=\tiny,
		backgroundcolor=\color[rgb]{.7,.7,.7}}}{}


\lstnewenvironment{lstlistingoutputtiny}{\lstset{basicstyle=\tiny\ttfamily,
		columns=flexible,
		breaklines=true,
		numbers=left,
		%stepsize=1,
		numberstyle=\tiny,
		backgroundcolor=\color[rgb]{.7,.7,.7}}}{}


% color-coded listings; replace those above 
\usepackage{xcolor}
\usepackage{minted}
\definecolor{listingbg}{rgb}{0.87,0.93,1}
\setminted[python]{
	frame=none,
	framesep=1mm,
	baselinestretch=1,
	bgcolor=listingbg,
	fontsize=\scriptsize,
	linenos,
	breaklines
	}


\usepackage[american]{babel}
\usepackage{csquotes}
\usepackage[style=apa, backend = biber]{biblatex}
\renewcommand*{\bibfont}{\tiny}


\usepackage{tikz}
\usetikzlibrary{shapes,arrows,matrix}
\usepackage{multicol}

\usepackage{subcaption}

\usepackage{booktabs}
\usepackage{graphicx}



\makeatletter
\setbeamertemplate{headline}{%
	\begin{beamercolorbox}[colsep=1.5pt]{upper separation line head}
	\end{beamercolorbox}
	\begin{beamercolorbox}{section in head/foot}
		\vskip2pt\insertnavigation{\paperwidth}\vskip2pt
	\end{beamercolorbox}%
	\begin{beamercolorbox}[colsep=1.5pt]{lower separation line head}
	\end{beamercolorbox}
}
\makeatother





\setbeamercolor{section in head/foot}{fg=normal text.bg, bg=structure.fg}


\newcommand{\instruction}[1]{\emph{\textcolor{gray}{[#1]}}}



\newcommand{\question}[1]{
	\begin{frame}[plain]
	\begin{columns}
		\column{.3\textwidth}
		\makebox[\columnwidth]{
			\includegraphics[width=\columnwidth,height=\paperheight,keepaspectratio]{mannetje.png}}
		\column{.7\textwidth}
		\large
		\textcolor{orange}{\textbf{\emph{#1}}}
	\end{columns}
\end{frame}}


\tikzstyle{block} = [rectangle, draw, fill=blue!20, 
text width=5em, text centered, rounded corners, minimum height=4em]
\tikzstyle{line} = [draw]
\tikzstyle{pijltje} = [draw, -latex']
\tikzstyle{cloud} = [draw, ellipse,fill=red!20, node distance=3cm,
minimum height=2em, text width=4em, text centered,]


\setbeamercovered{transparent}

\addbibresource{../../resources/literature.bib}
\graphicspath{{../../resources/img/}}


\begin{document}

\title[Big Data and Automated Content Analysis]{\textbf{Big Data and Automated Content Analysis (12EC)} 
\\Week 4: »Statistical Modelling and Machine learning«
\\Wednesday}
\author[Felicia Loecherbach]{Felicia Loecherbach\\ \footnotesize{f.loecherbach@uva.nl\\}}
\date{February 28, 2023}
\institute[UvA CW]{UvA RM Communication Science}


\begin{frame}{}
	\titlepage
\end{frame}

\begin{frame}{Today}
	\tableofcontents
\end{frame}

\begin{frame}{Some other things first}
	\begin{itemize}
		\item What should I do if I get stuck? 
		\begin{itemize}
			\item CCS Book 
			\item Documentations of packages (esp pandas, matplotlib)
			\item Online tutorials (e.g. freeCodeCamp videos)
			\item Other suggestions? 
		\end{itemize}
		\item What is the exam going to look like? 
		\begin{itemize}
			\item Two parts: Essay and data analysis task
			\item Both parts build on the things we discussed in class
			\item To prepare: Check again the slides and corresponding chapters in the CSS book, make sure you understand the exercises
			\item The coding part should be handed in as jupyter notebook-file, so make sure you know how to make those
		\end{itemize} 
	\end{itemize}

\end{frame}


\question{Everything clear from last week?}


\begin{frame}{Main points from last week}

\begin{alertblock}{I assume that by now, everybody knows:}
\begin{itemize}
\item how to get any data, in any shape, in all common file formats, into a structure usable for future analysis;
\item how to create a report in Jupyter Notebook using Markdown cells as well as matplotlib and seaborn (optionally: other visualization libraries).
\end{itemize}
\end{alertblock}
\end{frame}


\begin{frame}[standout]
This week, we will get a general overview of ML and statistical modelling. In Part II of the course, we will go more in-depth and specifically use these techniques in combination with textual data.
\end{frame}


\input{../../modules/machinelearning-intro/overview.tex}

\input{../../modules/machinelearning-intro/supervised.tex}

\input{../../modules/machinelearning-intro/unsupervised.tex}







\question{Any questions?}

\begin{frame}{One more thing...}
	\begin{itemize}
		\item The next few weeks we might be working more and more with retrieving your own data
		\item For this, APIs are very useful, but you often need to apply for access and it might take some time to get it or you need to create an account
		\item Starting application process to a few APIs, checking on Friday whether everyone has access 
		\begin{itemize}
			\item Wikipedia API (api.wikimedia.org)
			\item Gemeente Amsterdam API (api.data.amsterdam.nl)
			\item Schiphol PublicFlights API (developer.schiphol.nl)
			\item The Guardian Open Platform (open-platform.theguardian.com)
		\end{itemize}
	\end{itemize}
\end{frame}

\section{Next steps}

\begin{frame}[standout]
Make sure you understood all of today's concepts.

Re-read the chapters.

I prepared exercises to work on \emph{during} the Friday meeting (alone or in teams):
\large{\url{https://github.com/uvacw/teaching-bdaca/blob/main/12ec-course/week04/exercises/}}
\end{frame}


\begin{frame}[standout]
Take-home exam next week Friday!
\end{frame}





\begin{frame}[allowframebreaks,plain]
	\printbibliography
\end{frame}



\end{document}
