% !TeX document-id = {f19fb972-db1f-447e-9d78-531139c30778}
% !BIB program = biber

%\documentclass[handout]{beamer}
\documentclass[compress]{beamer}
\usepackage[T1]{fontenc}
\usetheme[block=fill,subsectionpage=progressbar,sectionpage=progressbar]{metropolis} 
\usepackage{graphicx}

\usepackage{wasysym}
\usepackage{etoolbox}
\usepackage[utf8]{inputenc}

\usepackage{pifont}

\usepackage{threeparttable}
\usepackage{subcaption}

\usepackage{tikz-qtree}
\setbeamercovered{still covered={\opaqueness<1->{5}},again covered={\opaqueness<1->{100}}}


\usepackage{listings}

\lstset{
	basicstyle=\scriptsize\ttfamily,
	columns=flexible,
	breaklines=true,
	numbers=left,
	%stepsize=1,
	numberstyle=\tiny,
	backgroundcolor=\color[rgb]{0.85,0.90,1}
}



\lstnewenvironment{lstlistingoutput}{\lstset{basicstyle=\footnotesize\ttfamily,
		columns=flexible,
		breaklines=true,
		numbers=left,
		%stepsize=1,
		numberstyle=\tiny,
		backgroundcolor=\color[rgb]{.7,.7,.7}}}{}


\lstnewenvironment{lstlistingoutputtiny}{\lstset{basicstyle=\tiny\ttfamily,
		columns=flexible,
		breaklines=true,
		numbers=left,
		%stepsize=1,
		numberstyle=\tiny,
		backgroundcolor=\color[rgb]{.7,.7,.7}}}{}



\usepackage[american]{babel}
\usepackage{csquotes}
\usepackage[style=apa, backend = biber]{biblatex}
\renewcommand*{\bibfont}{\tiny}


\usepackage{tikz}
\usetikzlibrary{shapes,arrows,matrix}
\usepackage{multicol}

\usepackage{subcaption}

\usepackage{booktabs}
\usepackage{graphicx}



\makeatletter
\setbeamertemplate{headline}{%
	\begin{beamercolorbox}[colsep=1.5pt]{upper separation line head}
	\end{beamercolorbox}
	\begin{beamercolorbox}{section in head/foot}
		\vskip2pt\insertnavigation{\paperwidth}\vskip2pt
	\end{beamercolorbox}%
	\begin{beamercolorbox}[colsep=1.5pt]{lower separation line head}
	\end{beamercolorbox}
}
\makeatother





\setbeamercolor{section in head/foot}{fg=normal text.bg, bg=structure.fg}



\newcommand{\question}[1]{
	\begin{frame}[plain]
	\begin{columns}
		\column{.3\textwidth}
		\makebox[\columnwidth]{
			\includegraphics[width=\columnwidth,height=\paperheight,keepaspectratio]{mannetje.png}}
		\column{.7\textwidth}
		\large
		\textcolor{orange}{\textbf{\emph{#1}}}
	\end{columns}
\end{frame}}


\tikzstyle{block} = [rectangle, draw, fill=blue!20, 
text width=5em, text centered, rounded corners, minimum height=4em]
\tikzstyle{line} = [draw]
\tikzstyle{pijltje} = [draw, -latex']
\tikzstyle{cloud} = [draw, ellipse,fill=red!20, node distance=3cm,
minimum height=2em, text width=4em, text centered,]


\setbeamercovered{transparent}

\addbibresource{../../resources/literature.bib}
\graphicspath{{../../resources/img/}}


\begin{document}

\title[Big Data and Automated Content Analysis]{\textbf{Big Data and Automated Content Analysis (12EC)} 
\\Week 5: »Processing textual data«
\\Wednesday}
\author[Damian Trilling]{Damian Trilling\\ \footnotesize{d.c.trilling@uva.nl, @damian0604 \\}}
\date{March 8, 2023}
\institute[UvA CW]{UvA RM Communication Science}


\begin{frame}{}
	\titlepage
\end{frame}

\begin{frame}{Today}
	\tableofcontents
\end{frame}



\question{Everything clear from last week?}


\begin{frame}{Main points from last week}

\begin{alertblock}{I assume that by now, everybody knows:}
\begin{itemize}
\item the relationship between ``traditional'' statistics and machine learning;
\item how to run unsupervised models with scikit-learn;
\item how to run supervised models with scikit-learn.
\end{itemize}
\end{alertblock}
\end{frame}


\begin{frame}[standout]
This week, we will get a general overview of working with textual data. Combining the knowledge from this week with last week gives you all blocks you need to do cool automated content analyses -- which we will start with next week.
\end{frame}


\section{Basic string operations}

\begin{frame}{Working with strings}
\begin{enumerate}[<+->]
\item string methods that every string has (\texttt{"hello".upper()})
\item functions that take a string as input (\texttt{len("hello")})
\item pandas column string methods (\texttt{df["somecolumn"].str.upper()})
\item applying string methods or functions to a pandas column (\texttt{df["somecolumn"].apply(len)} or \texttt{df["somecolumn"].apply(lambda x: x.upper()})
\end{enumerate}

\pause
For today, we assume that our data are a list of strings -- adapt accordingly for pandas.
\end{frame}


\begin{frame}[fragile]{An example says more than 1000 words\ldots}
\begin{minted}{python}
# probably read from text file(s) instead, you learned that already...
data = [ "I <b>really</b> liked this movie! It was great.  ", "  What an awful movie", "Awesome!!!"]

data_stripped = [e.strip() for e in data]
data_lower = [e.lower() for e in data_stripped]
data_clean = [e.replace("<b>",").replace("</b>",") for e in data_lower]

# or, more efficient, in one single step:
data_clean2 = [e.strip().lower().replace("<b>","").replace("</b>","") for e in data]
\end{minted}
\end{frame}



\begin{frame}[fragile]{Two examples says even more:}
\begin{minted}{python}
from string import punctuation

# punctuation is just the string '!"#$%&\'()*+,-./:;<=>?@[\\]^_`{|}~'

text = "This is a test! Let's get rid (of) punct&"

# we make a list of each character in the text but only if it is not
# a punctuation sign. The, we join the elements of the list directly
# to each other without anything between it ("")
cleantext = "".join([c for c in text if c not in punctuation])
\end{minted}
\end{frame}


\begin{frame}[fragile]{Combine both}
\begin{minted}{python}
from string import punctuation

def strip_punctuation(text):
    return "".join([c for c in text if c not in punctuation])

data_clean3 = [strip_punctuation(e).strip().lower()\
   .replace("<b>","").replace("</b>","") for e in data]
    
\end{minted}
\end{frame}




\begin{frame}{The toolbox at a glance}
  \footnotesize
\begin{block}{Slicing}
\texttt{mystring[2:5]} to get the characters with indices 2,3,4
\end{block}

\begin{block}{String methods}
\begin{itemize}
	\item \texttt{.lower()} returns lowercased string
	\item \texttt{.strip()} returns string without whitespace at beginning and end
	\item \texttt{.find("bla")} returns index of position of substring ``bla'' or -1 if not found
	\item \texttt{.replace("a","b")} returns string with "a" replaced by "b"
	\item \texttt{.count("bla")} counts how often substring ``bla'' occurs
        \item \texttt{.isdigit()} true if only numbers
          
\end{itemize}
Use tab completion for more!
\end{block}
\end{frame}




\begin{frame}[fragile]{From test to large-scale analysis: General approach}
1. Take a single string and test your idea
\begin{lstlisting}
t = "This is a test test test."
print(t.count("test"))
\end{lstlisting}
2a. You'd assume it to return 3. If so, scale it up:
\begin{lstlisting}
results = []
for t in listwithallmytexts:
    r = t.count("test")
    print(f"{t} contains the substring {r} times")
    results.append(r)
\end{lstlisting}

2b. If you \emph{only} need to get the list of results, a list comprehension is more elegant:
\begin{lstlisting}
results = [t.count("test") for t in listwithallmytexts]
\end{lstlisting}


\end{frame}


\begin{frame}[fragile]{General approach}
\Large

\textcolor{red}{Test on a single string, then make a for loop or list comprehension!}

\pause

\normalsize

\begin{alertblock}{Own functions}
If it gets more complex, you can write your own function and then use it in the list comprehension:
\begin{lstlisting}
def mycleanup(t):
   # do sth with string t here, create new string t2
   return t2
  
results = [mycleanup(t) for t in allmytexts]
\end{lstlisting}
\end{alertblock}
\end{frame}


\begin{frame}[fragile]{Pandas string methods as alternative}
If you select column with strings from a pandas dataframe, pandas offers a collection of string methods (via \texttt{.str.}) that largely mirror standard Python string methods:

\begin{lstlisting}
df['newcoloumnwithresults'] = df['columnwithtext'].str.count("bla")
\end{lstlisting} 


\pause

\begin{alertblock}{To pandas or not to pandas for text?}
Partly a matter of taste. 

Not-too-large dataset with a lot of extra columns? Advanced statistical analysis planned? Sounds like pandas.

It's mainly a lot of text? Wanna do some machine learning later on anyway? It's large and (potentially) messy? Doesn't sound like pandas is a good idea.
\end{alertblock}

\end{frame}



\section{Regular expressions}

\subsection{What is a regexp?}
\begin{frame}{Regular Expressions: What and why?}
\begin{block}{What is a regexp?}
\begin{itemize}
\item<1-> a \emph{very} widespread way to describe patterns in strings
\item<2-> Think of wildcards like {\tt{*}} or operators like {\tt{OR}}, {\tt{AND}} or {\tt{NOT}} in search strings: a regexp does the same, but is \emph{much} more powerful
\item<3-> You can use them in many editors (!), in the Terminal, in STATA \ldots and in Python
\end{itemize}
\end{block}
\end{frame}

\begin{frame}{A more powerful tool}
\begin{block}{An example}
\begin{itemize}
\item We want to remove everything but words from a tweet
\item We can do so by calling the \texttt{.replace()} method multiple times (for each unwanted character)
\item But we can better do this with a regular expression instead: \\
{\tt{ \lbrack \^{}a-zA-Z\rbrack}}  matches anything that is not a letter
\end{itemize}
\end{block}
\end{frame}

\begin{frame}{Basic regexp elements}
\begin{block}{Alternatives}
\begin{description}
\item[{\tt{\lbrack TtFf\rbrack}}] matches either T or t or F or f
\item[{\tt{Twitter|Facebook}}] matches either Twitter or Facebook
\item[{\tt{.}}] matches any character
\end{description}
\end{block}
\begin{block}{Repetition}<2->
\begin{description}
\item[{\tt{?}}] the expression before occurs 0 or 1 times
\item[{\tt{*}}] the expression before occurs 0 or more times
\item[{\tt{+}}] the expression before occurs 1 or more times
\end{description}
\end{block}
\end{frame}

\begin{frame}{regexp quizz}
\begin{block}{Which words would be matched?}
\tt
\begin{enumerate}
\item<1-> \lbrack Pp\rbrack ython
\item<2-> \lbrack A-Z\rbrack +
\item<3-> RT ?:? @\lbrack a-zA-Z0-9\rbrack +
\end{enumerate}
\end{block}
\end{frame}

\begin{frame}{What else is possible?}
See the table in the book!
\end{frame}

\subsection{Using a regexp in Python}
\begin{frame}{How to use regular expressions in Python}
\begin{block}{The module \texttt{re}*}
\begin{description}
\item<1->[{\tt{re.findall("\lbrack Tt\rbrack witter|\lbrack Ff\rbrack acebook",testo)}}] returns a list with all occurances of Twitter or Facebook in the string called {\tt{testo}}
\item<1->[{\tt{re.findall("\lbrack 0-9\rbrack +\lbrack a-zA-Z\rbrack +",testo)}}] returns a list with all words that start with one or more numbers followed by one or more letters in the string called {\tt{testo}}
\item<2->[{\tt{re.sub("\lbrack Tt\rbrack witter|\lbrack Ff\rbrack acebook","a social medium",testo)}}] returns a string in which all all occurances of Twitter or Facebook are replaced by "a social medium"
\end{description}
\end{block}

\tiny{Use the less-known but more powerful module \texttt{regex} instead to support all dialects used in the book}
\end{frame}


\begin{frame}[fragile]{How to use regular expressions in Python}
\begin{block}{The module re}
\begin{description}
\item<1->[{\tt{re.match(" +(\lbrack 0-9\rbrack +) of (\lbrack 0-9\rbrack +) points",line)}}] returns  \texttt{None} unless it \emph{exactly} matches the string \texttt{line}. If it does, you can access the part between () with the \texttt{.group()} method.
\end{description}
\end{block}

Example:
\begin{lstlisting}
line="             2 of 25 points"
result=re.match(" +([0-9]+) of ([0-9]+) points",line)
if result:
   print ("Your points:",result.group(1))
   print ("Maximum points:",result.group(2))
\end{lstlisting}
Your points: 2\\
Maximum points: 25
\end{frame}














\begin{frame}{Possible applications}
\begin{block}{Data preprocessing}
\begin{itemize}
\item Remove unwanted characters, words, \ldots
\item Identify \emph{meaningful} bits of text: usernames, headlines, where an article starts, \ldots
\item filter (distinguish relevant from irrelevant cases)
\end{itemize}
\end{block}
\end{frame}


\begin{frame}{Possible applications}
\begin{block}{Data analysis: Automated coding}
\begin{itemize}
\item Actors
\item Brands
\item links or other markers that follow a regular pattern
\item Numbers (!)
\end{itemize}
\end{block}
\end{frame}

\begin{frame}[fragile,plain]{Example 1: Counting actors}
\begin{minted}{python}
import re, csv
from glob import glob
counts1=[]
counts2=[]
filenames = glob("/home/damian/articles/*.txt")

for fn in filenames:
   with open(fn) as fi:
      artikel = fi.read()
      artikel = artikel.replace('\n',' ')
      
      counts1.append(len(re.findall('Israel.*(minister|politician.*|[Aa]uthorit)',artikel)))
      counts2.append(len(re.findall('[Pp]alest',artikel)))
      
output=zip(filenames, counts1, counts2)
with open("results.csv", mode='w',encoding="utf-8") as fo:
    writer = csv.writer(fo)
    writer.writerows(output)
\end{minted}
\end{frame}



\begin{frame}[fragile,plain]{Example 2: Parsing semi-structured data}
If your data look like this, you can loop over the lines and use regular expressions to extract the info you need!

\begin{lstlisting}
                              All Rights Reserved

                               2 of 200 DOCUMENTS

                                  De Telegraaf

                             21 maart 2014 vrijdag

Brussel bereikt akkoord  aanpak probleembanken;
ECB krijgt meer in melk te brokkelen

SECTION: Finance; Blz. 24
LENGTH: 660 woorden

BRUSSEL   Europa heeft gisteren op de valreep een akkoord bereikt 
over een saneringsfonds voor banken. Daarmee staat de laatste
\end{lstlisting}

\end{frame}



\begin{frame}{Practice yourself!}
	Take some time to write some regular expressions.
	Write a script that
\begin{itemize}
	\item extracts URLS form a list of strings
	\item removes everything that is not a letter or number from a list of strings
\end{itemize}
(first develop it for a single string, then scale up)

More tips:
\huge{\url{http://www.pyregex.com/}}
\end{frame}



\section{The Bag-of-Words (BOW) Representation}













\question{Any questions?}

\section{Next steps}

\begin{frame}[standout]
Make sure you understood all of today's concepts.

Re-read the chapters.

I prepared exercises to work on \emph{during} the Friday meeting (alone or in teams):
\large{\url{https://github.com/uvacw/teaching-bdaca/blob/main/12ec-course/week05/exercises/}}


\textbf{Take-home exam on Friday!}

\end{frame}





\begin{frame}[allowframebreaks,plain]
	\printbibliography
\end{frame}



\end{document}
