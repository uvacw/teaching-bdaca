% !TeX document-id = {f19fb972-db1f-447e-9d78-531139c30778}
% !BIB program = biber

%\documentclass[handout]{beamer}
%\documentclass[compress]{beamer}
\usepackage[T1]{fontenc}
\usetheme[block=fill,subsectionpage=progressbar,sectionpage=progressbar]{metropolis} 
\usepackage{graphicx}

\usepackage{wasysym}
\usepackage{etoolbox}
\usepackage[utf8]{inputenc}

\usepackage{pifont}

\usepackage{threeparttable}
\usepackage{subcaption}

\usepackage{tikz-qtree}
\usepackage{neuralnetwork}

\setbeamercovered{still covered={\opaqueness<1->{5}},again covered={\opaqueness<1->{100}}}


\usepackage{listings}

\lstset{
	basicstyle=\scriptsize\ttfamily,
	columns=flexible,
	breaklines=true,
	numbers=left,
	%stepsize=1,
	numberstyle=\tiny,
	backgroundcolor=\color[rgb]{0.85,0.90,1}
}



\lstnewenvironment{lstlistingoutput}{\lstset{basicstyle=\footnotesize\ttfamily,
		columns=flexible,
		breaklines=true,
		numbers=left,
		%stepsize=1,
		numberstyle=\tiny,
		backgroundcolor=\color[rgb]{.7,.7,.7}}}{}


\lstnewenvironment{lstlistingoutputtiny}{\lstset{basicstyle=\tiny\ttfamily,
		columns=flexible,
		breaklines=true,
		numbers=left,
		%stepsize=1,
		numberstyle=\tiny,
		backgroundcolor=\color[rgb]{.7,.7,.7}}}{}


% color-coded listings; replace those above 
\usepackage{xcolor}
\usepackage{minted}
\definecolor{listingbg}{rgb}{0.87,0.93,1}
\setminted[python]{
	frame=none,
	framesep=1mm,
	baselinestretch=1,
	bgcolor=listingbg,
	fontsize=\scriptsize,
	linenos,
	breaklines
	}


\usepackage[american]{babel}
\usepackage{csquotes}
\usepackage[style=apa, backend = biber]{biblatex}
\renewcommand*{\bibfont}{\tiny}


\usepackage{tikz}
\usetikzlibrary{shapes,arrows,matrix}
\usepackage{multicol}

\usepackage{subcaption}

\usepackage{booktabs}
\usepackage{graphicx}



\makeatletter
\setbeamertemplate{headline}{%
	\begin{beamercolorbox}[colsep=1.5pt]{upper separation line head}
	\end{beamercolorbox}
	\begin{beamercolorbox}{section in head/foot}
		\vskip2pt\insertnavigation{\paperwidth}\vskip2pt
	\end{beamercolorbox}%
	\begin{beamercolorbox}[colsep=1.5pt]{lower separation line head}
	\end{beamercolorbox}
}
\makeatother





\setbeamercolor{section in head/foot}{fg=normal text.bg, bg=structure.fg}


\newcommand{\instruction}[1]{\emph{\textcolor{gray}{[#1]}}}



\newcommand{\question}[1]{
	\begin{frame}[plain]
	\begin{columns}
		\column{.3\textwidth}
		\makebox[\columnwidth]{
			\includegraphics[width=\columnwidth,height=\paperheight,keepaspectratio]{mannetje.png}}
		\column{.7\textwidth}
		\large
		\textcolor{orange}{\textbf{\emph{#1}}}
	\end{columns}
\end{frame}}


\tikzstyle{block} = [rectangle, draw, fill=blue!20, 
text width=5em, text centered, rounded corners, minimum height=4em]
\tikzstyle{line} = [draw]
\tikzstyle{pijltje} = [draw, -latex']
\tikzstyle{cloud} = [draw, ellipse,fill=red!20, node distance=3cm,
minimum height=2em, text width=4em, text centered,]


\setbeamercovered{transparent}

\addbibresource{../../resources/literature.bib}
\graphicspath{{../../resources/img/}}


\begin{document}

\title[Big Data and Automated Content Analysis]{\textbf{Big Data and Automated Content Analysis (12EC)} 
\\Week 5: »Processing textual data«
\\Wednesday}
\author[Felicia Loecherbach]{Felicia Loecherbach\\ \footnotesize{f.loecherbach@uva.nl\\}}
\date{March 8, 2023}
\institute[UvA CW]{UvA RM Communication Science}


\begin{frame}{}
	\titlepage
\end{frame}

\begin{frame}{Today}
	\tableofcontents
\end{frame}



\question{Everything clear from last week?}


\begin{frame}{Main points from last week}

\begin{alertblock}{I assume that by now, everybody knows:}
\begin{itemize}
\item the relationship between ``traditional'' statistics and machine learning;
\item how to run unsupervised models with scikit-learn;
\item how to run supervised models with scikit-learn.
\end{itemize}
\end{alertblock}
\end{frame}


\begin{frame}[standout]
This week, we will get a general overview of working with textual data. Combining the knowledge from this week with last week gives you all blocks you need to do cool automated content analyses -- which we will start with next week.
\end{frame}


\input{../../modules/working-with-text/basic-string-operations.tex}

\section{Regular expressions}

\subsection{What is a regexp?}
\begin{frame}{Regular Expressions: What and why?}
\begin{block}{What is a regexp?}
\begin{itemize}
\item<1-> a \emph{very} widespread way to describe patterns in strings
\item<2-> Think of wildcards like {\tt{*}} or operators like {\tt{OR}}, {\tt{AND}} or {\tt{NOT}} in search strings: a regexp does the same, but is \emph{much} more powerful
\item<3-> You can use them in many editors (!), in the Terminal, in STATA \ldots and in Python
\end{itemize}
\end{block}
\end{frame}

\begin{frame}{A more powerful tool}
\begin{block}{An example}
\begin{itemize}
\item We want to remove everything but words from a tweet
\item We can do so by calling the \texttt{.replace()} method multiple times (for each unwanted character)
\item We can do so with a join+list comprehension:\\  \texttt{"".join([c for c in tweet if c not in listwithunwantedcharacters])}
\item But we can also use a regular expression instead: \\
{\tt{ \lbrack \^{}a-zA-Z\rbrack}}  matches anything that is not a letter
\end{itemize}
\end{block}
\end{frame}

\begin{frame}{Basic regexp elements}
\begin{block}{Alternatives}
\begin{description}
\item[{\tt{\lbrack TtFf\rbrack}}] matches either T or t or F or f
\item[{\tt{Twitter|Facebook}}] matches either Twitter or Facebook
\item[{\tt{.}}] matches any character
\end{description}
\end{block}
\begin{block}{Repetition}<2->
\begin{description}
\item[{\tt{?}}] the expression before occurs 0 or 1 times
\item[{\tt{*}}] the expression before occurs 0 or more times
\item[{\tt{+}}] the expression before occurs 1 or more times
\end{description}
\end{block}
\end{frame}

\begin{frame}{regexp quizz}
\begin{block}{Which words would be matched?}
\tt
\begin{enumerate}
\item<1-> \lbrack Pp\rbrack ython
\item<2-> \lbrack A-Z\rbrack +
\item<3-> RT ?:? @\lbrack a-zA-Z0-9\rbrack +
\end{enumerate}
\end{block}
\end{frame}

\begin{frame}{What else is possible?}
See the table in the book!
\end{frame}

\subsection{Using a regexp in Python}
\begin{frame}{How to use regular expressions in Python}
\begin{block}{The module \texttt{re}*}
\begin{description}
\item<1->[{\tt{re.findall("\lbrack Tt\rbrack witter|\lbrack Ff\rbrack acebook",testo)}}] returns a list with all occurances of Twitter or Facebook in the string called {\tt{testo}}
\item<1->[{\tt{re.findall("\lbrack 0-9\rbrack +\lbrack a-zA-Z\rbrack +",testo)}}] returns a list with all words that start with one or more numbers followed by one or more letters in the string called {\tt{testo}}
\item<2->[{\tt{re.sub("\lbrack Tt\rbrack witter|\lbrack Ff\rbrack acebook","a social medium",testo)}}] returns a string in which all all occurances of Twitter or Facebook are replaced by "a social medium"
\end{description}
\end{block}

\tiny{Use the less-known but more powerful module \texttt{regex} instead to support all dialects used in the book}
\end{frame}


\begin{frame}[fragile]{How to use regular expressions in Python}
\begin{block}{The module re}
\begin{description}
\item<1->[{\tt{re.match(" +(\lbrack 0-9\rbrack +) of (\lbrack 0-9\rbrack +) points",line)}}] returns  \texttt{None} unless it \emph{exactly} matches the string \texttt{line}. If it does, you can access the part between () with the \texttt{.group()} method.
\end{description}
\end{block}

Example:
\begin{lstlisting}
line="             2 of 25 points"
result=re.match(" +([0-9]+) of ([0-9]+) points",line)
if result:
   print ("Your points:",result.group(1))
   print ("Maximum points:",result.group(2))
\end{lstlisting}
Your points: 2\\
Maximum points: 25
\end{frame}














\begin{frame}{Possible applications}
\begin{block}{Data preprocessing}
\begin{itemize}
\item Remove unwanted characters, words, \ldots
\item Identify \emph{meaningful} bits of text: usernames, headlines, where an article starts, \ldots
\item filter (distinguish relevant from irrelevant cases)
\end{itemize}
\end{block}
\end{frame}


\begin{frame}{Possible applications}
\begin{block}{Data analysis: Automated coding}
\begin{itemize}
\item Actors
\item Brands
\item links or other markers that follow a regular pattern
\item Numbers (!)
\end{itemize}
\end{block}
\end{frame}

\begin{frame}[fragile,plain]{Example 1: Counting actors}
\begin{minted}{python}
import re, csv
from glob import glob
counts1=[]
counts2=[]
filenames = glob("/home/felicia/articles/*.txt")

for fn in filenames:
   with open(fn) as fi:
      artikel = fi.read()
      artikel = artikel.replace('\n',' ')
      
      counts1.append(len(re.findall('Israel.*(minister|politician.*|[Aa]uthorit)',artikel)))
      counts2.append(len(re.findall('[Pp]alest',artikel)))
      
output=zip(filenames, counts1, counts2)
with open("results.csv", mode='w',encoding="utf-8") as fo:
    writer = csv.writer(fo)
    writer.writerows(output)
\end{minted}
\end{frame}



\begin{frame}[fragile,plain]{Example 2: Parsing semi-structured data}
If your data look like this, you can loop over the lines and use regular expressions to extract the info you need!

\begin{lstlisting}
                              All Rights Reserved

                               2 of 200 DOCUMENTS

                                  De Telegraaf

                             21 maart 2014 vrijdag

Brussel bereikt akkoord  aanpak probleembanken;
ECB krijgt meer in melk te brokkelen

SECTION: Finance; Blz. 24
LENGTH: 660 woorden

BRUSSEL   Europa heeft gisteren op de valreep een akkoord bereikt 
over een saneringsfonds voor banken. Daarmee staat de laatste
\end{lstlisting}

\end{frame}



\begin{frame}{Practice yourself!}
	Take some time to write some regular expressions.
	Write a script that
\begin{itemize}
	\item extracts URLS form a list of strings
	\item removes everything that is not a letter or number from a list of strings
\end{itemize}
(first develop it for a single string, then scale up)

More tips:
\huge{\url{http://www.pyregex.com/}}
\end{frame}



\section[The BOW]{The bag-of-words (BOW) model}

\subsection{General idea}

\begin{frame}[fragile]{A text as a collections of word}

Let us represent a string 
\begin{lstlisting}
t = "This this is is is a test test test"
\end{lstlisting}
like this:\\
\begin{lstlisting}
from collections import Counter
print(Counter(t.split()))
\end{lstlisting}
\begin{lstlistingoutput}
Counter({'is': 3, 'test': 3, 'This': 1, 'this': 1, 'a': 1})
\end{lstlistingoutput}

\pause 
Compared to the original string, this representation
\begin{itemize}
	\item is less repetitive
	\item preserves word frequencies
	\item but does \emph{not} preserve word order
	\item can be interpreted as a vector to calculate with (!!!)
\end{itemize}

\tiny{\emph{Of course, still a lot of stuff to fine-tune\ldots}  (for example, This/this)}
\end{frame}



\begin{frame}{From vector to matrix}
If we do this for multiple texts, we can arrange the vectors in a table.

t1 = "This this is is is a test test test" \newline
t2 = "This is an example"

\begin{tabular}{| c|c|c|c|c|c|c|c|}
	\hline
	& a & an & example & is & this & This & test \\
	\hline
	\emph{t1} & 1 & 0 & 0 & 3 & 1 & 1 & 3 \\
	\emph{t2} &0 & 1 & 1 & 1 & 0 & 1 & 0 \\
	\hline
\end{tabular}
\end{frame}


\question{What can you do with such a matrix? Why would you want to represent a collection of texts in such a way?}


\begin{frame}{The cell entries: raw counts versus tf$\cdot$idf scores}
\begin{itemize}
	\item In the example, we entered simple counts (the ``term frequency'')
\end{itemize}
\end{frame}

\question{But are all terms equally important?}


\begin{frame}{The cell entries: raw counts versus tf$\cdot$idf scores}
	\begin{itemize}
		\item In the example, we entered simple counts (the ``term frequency'')
		\item But does a word that occurs in almost all documents contain much information?
		\item And isn't the presence of a word that occurs in very few documents a pretty strong hint?
		\item<2-> \textbf{Solution: Weigh by \emph{the number of documents in which the term occurs at least once) (the ``document frequency'')}} 
	\end{itemize}
\onslide<3->{
$\Rightarrow$ we multiply the ``term frequency'' (tf) by the inverse document frequency (idf)

\tiny{(usually with some additional logarithmic transformation and normalization applied, see \url{https://scikit-learn.org/stable/modules/generated/sklearn.feature_extraction.text.TfidfTransformer.html})}
}
\end{frame}

\begin{frame}{Is tf$\cdot$idf always better?}
It depends.

\begin{itemize}
	\item Ultimately, it's an empirical question which works better ($\rightarrow$ weeks on machine learning)
	\item In many scenarios,  ``discounting'' too frequent words and ``boosting'' rare words makes a lot of sense (most frequent words in a text can be highly un-informative)
	\item Beauty of raw tf counts, though: interpretability + describes document in itself, not in relation to other documents
\end{itemize}
\end{frame}


\begin{frame}{Internal representations}
\begin{block}{Sparse vs dense matrices}
\begin{itemize}
	\item Most terms are \emph{not} contained in a given document
	\item $\rightarrow$ tens of thousands of columns (terms), and one row per document
	\item Filling all cells is inefficient \emph{and} can make the matrix too large to fit in memory (!!!)
	\item Solution: store only non-zero values with their coordinates! (sparse matrix)
	\item dense matrix (or dataframes) not advisable, only for toy examples
\end{itemize}
\end{block}
\end{frame}



\begin{frame}{Internal representations}
  \begin{alertblock}{Little over-generalizing R vs Python remark ;-)}
    Among many R users, it is common to manually inspect document-term matrices, and many operations are done directly on them. In Python, they are more commonly seen as a means to an end (mostly, as input for machine learning).
          
    Many R modules\footnote{Things have become a bit better recently} convert to dense matrices: really problematic for larger datasets!
  \end{alertblock}
	
\end{frame}






\subsection{A cleaner BOW representation}

\begin{frame}{Room for improvement}
\begin{description}
	\item[tokenization] How do we (best) split a sentence into tokens (terms, words)?
	\item[pruning] How can we remove unneccessary words?
	\item[lemmatization] How can we make sure that slight variations of the same word are not counted differently?

\end{description}
\end{frame}

\subsubsection{Better tokenization}

\begin{frame}[fragile]{OK, good enough, perfect?}
\begin{block}{.split()}
\begin{itemize}
	\item space $\rightarrow$ new word
	\item no further processing whatsoever
	\item thus, only works well if we do a preprocessing outselves (e.g., remove punctuation)
\end{itemize}
\end{block}
\begin{lstlisting}
docs = ["This is a text",  "I haven't seen John's derring-do. Second sentence!"]
tokens = [d.split() for d in docs]
\end{lstlisting}
\begin{lstlistingoutputtiny}
[['This', 'is', 'a', 'text'], ['I', "haven't", 'seen', "John's", 'derring-do.', 'Second', 'sentence!']]
\end{lstlistingoutputtiny}
\end{frame}


\begin{frame}{OK, good enough, perfect?}
  \begin{block}{Tokenizers from the NLTK pacakge}
    \begin{itemize}
    \item multiple improved tokenizers that can be used instead of .split()
    \item e.g., Treebank tokenizer:
      \begin{itemize}
      \item split standard contractions ("don't")
      \item deals with punctuation
      \item BUT: Assumes lists of \emph{sentences}.
      \end{itemize}
    \item Solution: Build an own (combined) tokenizer (next slide)!
    \end{itemize}
  \end{block}
\end{frame}


\begin{frame}[fragile]{OK, good enough, perfect?}
\begin{minted}{python}
import nltk
import re

class MyTokenizer:
    def tokenize(self, text):
        tokenizer = nltk.tokenize.TreebankWordTokenizer()
        result = []
        word = r"\w"
        for sent in nltk.sent_tokenize(text):
            tokens = tokenizer.tokenize(sent)    
            tokens = [t for t in tokens 
                      if re.search(word, t)]
            result += tokens
        return result
        
mytokenizer = MyTokenizer()
tokens = [mytokenizer.tokenize(d) for d in docs]

\end{minted}

\begin{lstlistingoutputtiny}
[['This', 'is', 'a', 'text'], ['I', 'have', "n't", 'seen', 'John', "'s", 'derring-do', 'Second', 'sentence']]
\end{lstlistingoutputtiny}
\end{frame}


\question{Can you (try to) explain the code?}




\begin{frame}[standout]
OK, so we can tokenize with a list comprehension (and that's often a good idea!). But what if we want to \emph{directly} get a DTM instead of lists of tokens?
\end{frame}


\begin{frame}[fragile]{OK, good enough, perfect?}
  \begin{block}{scikit-learn's CountVectorizer (default settings)}
    \begin{itemize}
    \item applies lowercasing
    \item deals with punctuation etc. itself
    \item minimum word length $>1$
    \item more technically, tokenizes using this regular expression: \texttt{r"(?u)\textbackslash b\textbackslash w\textbackslash w+\textbackslash b"} \footnote{?u = support unicode, \textbackslash b = word boundary}
    \end{itemize}
  \end{block}
\begin{lstlisting}
from sklearn.feature_extraction.text import CountVectorizer
cv = CountVectorizer()
dtm_sparse = cv.fit_transform(docs)
\end{lstlisting}
\end{frame}


\begin{frame}{OK, good enough, perfect?}
  \begin{block}{CountVectorizer supports more}
	  \begin{itemize}
	  \item stopword removal
	  \item custom regular expression
	  \item or even using an external tokenizer
	  \item ngrams instead of unigrams
	  \end{itemize}
  \end{block}
  \tiny{see \url{https://scikit-learn.org/stable/modules/generated/sklearn.feature\_extraction.text.CountVectorizer.html}}

\pause
\begin{alertblock}{Best of both worlds}
  \textbf{Use the Count vectorizer with the custom NLTK-based external tokenizer we created before!}
  \texttt{cv = CountVectorizer(tokenizer=mytokenizer.tokenize)}
\end{alertblock}
\end{frame}



\subsubsection{Stopword removal}



\begin{frame}{Stopword removal}
	\begin{block}{What are stopwords?}
		\begin{itemize}
			\item Very frequent words with little inherent meaning
			\item \texttt{the, a, he, she, \ldots}
			\item context-dependent: if you are interested in gender, \texttt{he} and \texttt{she} are no stopwords. 
			\item Many existing lists as basis
		\end{itemize}
	\end{block}

When using the CountVectorizer, we can simply provide a stopword list. 

But we can also remove stopwords ``by hand'' of course using either a for loop (like we did for punctuation removal) or by modifying the tokennizer (try it!).
\end{frame}




\subsubsection{Pruning}

\begin{frame}{General idea}
\begin{itemize}
	\item Idea behind both stopword removal and tf$\cdot$idf: too frequent words are uninformative
	\item<2-> (possible) downside stopword removal: a priori list, does not take empirical frequencies in dataset into account
	\item<3-> (possible) downside tf$\cdot$idf: does not reduce number of features
\end{itemize}

\onslide<4->{Pruning: remove all features (tokens) that occur in less than X or more than X of the documents}
\end{frame}

\begin{frame}[fragile, plain]
CountVectorizer, only stopword removal
\begin{lstlisting}
from sklearn.feature_extraction.text import CountVectorizer, TfidfVectorizer
myvectorizer = CountVectorizer(stop_words=mystopwords)
\end{lstlisting}

CountVectorizer, other tokenization, stopword removal (pay attention that stopword list uses same tokenization!):
\begin{lstlisting}
myvectorizer = CountVectorizer(tokenizer = TreebankWordTokenizer().tokenize, stop_words=mystopwords)
\end{lstlisting}

Additionally remove words that occur in more than 75\% or less than $n=2$ documents:
\begin{lstlisting}
myvectorizer = CountVectorizer(tokenizer = TreebankWordTokenizer().tokenize, stop_words=mystopwords, max_df=.75, min_df=2)
\end{lstlisting}

All togehter: tf$\cdot$idf, explicit stopword removal, pruning
\begin{lstlisting}
myvectorizer = TfidfVectorizer(tokenizer = TreebankWordTokenizer().tokenize, stop_words=mystopwords, max_df=.75, min_df=2)
\end{lstlisting}


\end{frame}


\question{What is ``best''? Which (combination of) techniques to use, and how to decide?}



\subsubsection{Stemming and lemmatization}


\begin{frame}[fragile]{Stemming and lemmatization}
\begin{itemize}
\item Stemming: reduce words to its stem by removing last part (drinking $\rightarrow$ drink)
\item Lemmatization: find word that you would need to look up in a dictionary (drinking $\rightarrow$ drink, but also went $\rightarrow$ go)
\item stemming is simpler than lemmatization
\item lemmatization often better
\end{itemize}
\pause

Example below: tokenization and lemmatization with \texttt{spacy} in one go:
\begin{lstlisting}
import spacy
nlp = spacy.load('en')   # potentially you need to install the language model first
lemmatized_tokens = [[token.lemma_  for token in nlp(doc)] for doc in docs]
\end{lstlisting}
\begin{lstlistingoutputtiny}
[['this', 'be', 'a', 'text'], ['-PRON-', 'have', 'not', 'see', 'John', "'s", 'derring', '-', 'do', '.', 'second', 'sentence', '!']]
\end{lstlistingoutputtiny}
\end{frame}



\subsection{The order of preprocessing steps}

\begin{frame}{Option 1}
\begin{block}{Preprocessing only through Vectorizer}
``Just use CountVectorizer or Tfidfvectorizer with the appropriate options.''	
\begin{itemize}
	\item pro: No double work, efficient if your main goal is a sparse matrix (for ML?) anyway
	\item con: you cannot ``see'' the preprocessed texts
\end{itemize}
\end{block}
\end{frame}

\begin{frame}[fragile]{Option 2}
	\begin{block}{Extensive preprocessing without Vectorizer}
``Remove stopwords, punctuation etc. and store in a string with spaces''

\begin{lstlisting}
cleaneddocs = [" ".join(re.findall(r"\w\w+", d)).lower() for d in docs]
cleaneddocswithoutstopwords = [" ".join([w for w in d.split() if w not in mystopwords]) for d in cleaneddocs]
\end{lstlisting}
\begin{lstlistingoutputtiny}
['this is text', 'haven seen john derring do second sentence']
['text', 'seen john derring second sentence']	
\end{lstlistingoutputtiny}
{\tiny{Yes, this list comprehension looks scary -- you can make a more elaborate for loop instead}}
	
\begin{itemize}
	\item pro: you can read (and store!) the preprocessed docs
	\item pro: even the most stupid vectorizer (or wordcloud tool) can split the resulting string later on
	\item con: potentially double work (for you \emph{and} the computer)
\end{itemize}
\end{block}
\end{frame}


\question{How would you do it?}

\begin{frame}[plain]
Sometimes, I go for Option 2 because
\begin{itemize}
	\item I like to inspect a sample of the documents
	\item I can re-use the cleaned docs irrespective of the Vectorizer
\end{itemize}

But at other times, I opt of Option 1 instead because
\begin{itemize}
	\item I want to systematically compare the effect of different choices in a machine learning pipeline (then I can simply vary the vectorizer instead of the data)
	\item I want to use techniques that are geared towards little or no preprocessing (deep learning)
\end{itemize}

\end{frame}


\subsection{How further?}


\begin{frame}{Main takeaway}

\begin{itemize}
	\item It matters how you transform your text into numbers (``vectorization'').
	\item Preprocessing matters, be able to make informed choices.
	\item Keep this in mind when we will discuss Machine Learning! It will come back throughout Part II!
\end{itemize}

\begin{itemize}
	\item Once you vectorized your texts, you can do all kinds of calculations (random example: get the cosine similarity between two texts)
\end{itemize}

\end{frame}


\begin{frame}{More NLP}
\begin{description}
	\item[$n$-grams] Consider using $n$-grams instead of unigrams
	\item[collocations]  $n$grams that appear more frequently than expected
	\item[POS-tagging] grammatical function (``part-of-speach'') of tokens
	\item[NER] named entity recognition (persons, organizations, locations)
\end{description}
\end{frame}

\begin{frame}{More NLP}
I \textbf{really} recommend looking into spacy (\url{https://spacy.io}) for advanced natural language processing, such as part-of-speech-tagging and named entity recogntion.
\end{frame}














\question{Any questions?}

\section{Next steps}

\begin{frame}[standout]
Make sure you understood all of today's concepts.

Re-read the chapters.

I prepared exercises to work on \emph{during} the Friday meeting (alone or in teams):
\large{\url{https://github.com/uvacw/teaching-bdaca/blob/main/12ec-course/week05/exercises/}}


\textbf{Take-home exam on Friday!}

\end{frame}





\begin{frame}[allowframebreaks,plain]
	\printbibliography
\end{frame}



\end{document}
