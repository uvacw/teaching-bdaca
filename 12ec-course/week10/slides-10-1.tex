\documentclass[compress]{beamer}
% !TeX document-id = {f19fb972-db1f-447e-9d78-531139c30778}
% !BIB program = biber

%\documentclass[handout]{beamer}
%\documentclass[compress]{beamer}
\usepackage[T1]{fontenc}
\usetheme[block=fill,subsectionpage=progressbar,sectionpage=progressbar]{metropolis} 
\usepackage{graphicx}

\usepackage{wasysym}
\usepackage{etoolbox}
\usepackage[utf8]{inputenc}

\usepackage{pifont}

\usepackage{threeparttable}
\usepackage{subcaption}

\usepackage{tikz-qtree}
\usepackage{neuralnetwork}

\setbeamercovered{still covered={\opaqueness<1->{5}},again covered={\opaqueness<1->{100}}}


\usepackage{listings}

\lstset{
	basicstyle=\scriptsize\ttfamily,
	columns=flexible,
	breaklines=true,
	numbers=left,
	%stepsize=1,
	numberstyle=\tiny,
	backgroundcolor=\color[rgb]{0.85,0.90,1}
}



\lstnewenvironment{lstlistingoutput}{\lstset{basicstyle=\footnotesize\ttfamily,
		columns=flexible,
		breaklines=true,
		numbers=left,
		%stepsize=1,
		numberstyle=\tiny,
		backgroundcolor=\color[rgb]{.7,.7,.7}}}{}


\lstnewenvironment{lstlistingoutputtiny}{\lstset{basicstyle=\tiny\ttfamily,
		columns=flexible,
		breaklines=true,
		numbers=left,
		%stepsize=1,
		numberstyle=\tiny,
		backgroundcolor=\color[rgb]{.7,.7,.7}}}{}


% color-coded listings; replace those above 
\usepackage{xcolor}
\usepackage{minted}
\definecolor{listingbg}{rgb}{0.87,0.93,1}
\setminted[python]{
	frame=none,
	framesep=1mm,
	baselinestretch=1,
	bgcolor=listingbg,
	fontsize=\scriptsize,
	linenos,
	breaklines
	}


\usepackage[american]{babel}
\usepackage{csquotes}
\usepackage[style=apa, backend = biber]{biblatex}
\renewcommand*{\bibfont}{\tiny}


\usepackage{tikz}
\usetikzlibrary{shapes,arrows,matrix}
\usepackage{multicol}

\usepackage{subcaption}

\usepackage{booktabs}
\usepackage{graphicx}



\makeatletter
\setbeamertemplate{headline}{%
	\begin{beamercolorbox}[colsep=1.5pt]{upper separation line head}
	\end{beamercolorbox}
	\begin{beamercolorbox}{section in head/foot}
		\vskip2pt\insertnavigation{\paperwidth}\vskip2pt
	\end{beamercolorbox}%
	\begin{beamercolorbox}[colsep=1.5pt]{lower separation line head}
	\end{beamercolorbox}
}
\makeatother





\setbeamercolor{section in head/foot}{fg=normal text.bg, bg=structure.fg}


\newcommand{\instruction}[1]{\emph{\textcolor{gray}{[#1]}}}



\newcommand{\question}[1]{
	\begin{frame}[plain]
	\begin{columns}
		\column{.3\textwidth}
		\makebox[\columnwidth]{
			\includegraphics[width=\columnwidth,height=\paperheight,keepaspectratio]{mannetje.png}}
		\column{.7\textwidth}
		\large
		\textcolor{orange}{\textbf{\emph{#1}}}
	\end{columns}
\end{frame}}


\tikzstyle{block} = [rectangle, draw, fill=blue!20, 
text width=5em, text centered, rounded corners, minimum height=4em]
\tikzstyle{line} = [draw]
\tikzstyle{pijltje} = [draw, -latex']
\tikzstyle{cloud} = [draw, ellipse,fill=red!20, node distance=3cm,
minimum height=2em, text width=4em, text centered,]


\setbeamercovered{transparent}

\addbibresource{../../resources/literature.bib}
\graphicspath{{../../resources/img/}}


\begin{document}

\title[Big Data and Automated Content Analysis]{\textbf{Big Data and Automated Content Analysis (12EC)} 
\\Week 10: »Web Scraping«
\\Wednesday}
\author[Felicia Loecherbach]{Felicia Loecherbach\\ \footnotesize{f.loecherbach@uva.nl \\}}
\date{April 19, 2023}
\institute[UvA CW]{UvA RM Communication Science}


\begin{frame}{}
	\titlepage
\end{frame}

\begin{frame}{Today}
	\tableofcontents
\end{frame}



\question{Everything clear from last week?}


\begin{frame}{Main points from last week}

\begin{alertblock}{I assume that by now, everybody knows:}
\begin{itemize}
\item how to work with textual data;
\item and in particular (for this week) the concept of regular expressions
\end{itemize}
\end{alertblock}
\end{frame}


\begin{frame}[standout]
This week, we will learn how to gather non-structured online data.
\end{frame}


\input{../../modules/scraping/scraping.tex}










\question{Any questions?}

\section{Next steps}

\begin{frame}[standout]
You will write your own scraper. \textcolor{orange}{Prepare} by choosing a website that you want to scrape. It is advisable to select a site without things like cookiewalls or logins.
\large{\url{https://github.com/uvacw/teaching-bdaca/blob/main/12ec-course/week10/exercises/}}
\end{frame}





\begin{frame}[allowframebreaks,plain]
	\printbibliography
\end{frame}



\end{document}
