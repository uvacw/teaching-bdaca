\documentclass[compress]{beamer}
% !TeX document-id = {f19fb972-db1f-447e-9d78-531139c30778}
% !BIB program = biber

%\documentclass[handout]{beamer}
%\documentclass[compress]{beamer}
\usepackage[T1]{fontenc}
\usetheme[block=fill,subsectionpage=progressbar,sectionpage=progressbar]{metropolis} 
\usepackage{graphicx}

\usepackage{wasysym}
\usepackage{etoolbox}
\usepackage[utf8]{inputenc}

\usepackage{pifont}

\usepackage{threeparttable}
\usepackage{subcaption}

\usepackage{tikz-qtree}
\usepackage{neuralnetwork}

\setbeamercovered{still covered={\opaqueness<1->{5}},again covered={\opaqueness<1->{100}}}


\usepackage{listings}

\lstset{
	basicstyle=\scriptsize\ttfamily,
	columns=flexible,
	breaklines=true,
	numbers=left,
	%stepsize=1,
	numberstyle=\tiny,
	backgroundcolor=\color[rgb]{0.85,0.90,1}
}



\lstnewenvironment{lstlistingoutput}{\lstset{basicstyle=\footnotesize\ttfamily,
		columns=flexible,
		breaklines=true,
		numbers=left,
		%stepsize=1,
		numberstyle=\tiny,
		backgroundcolor=\color[rgb]{.7,.7,.7}}}{}


\lstnewenvironment{lstlistingoutputtiny}{\lstset{basicstyle=\tiny\ttfamily,
		columns=flexible,
		breaklines=true,
		numbers=left,
		%stepsize=1,
		numberstyle=\tiny,
		backgroundcolor=\color[rgb]{.7,.7,.7}}}{}


% color-coded listings; replace those above 
\usepackage{xcolor}
\usepackage{minted}
\definecolor{listingbg}{rgb}{0.87,0.93,1}
\setminted[python]{
	frame=none,
	framesep=1mm,
	baselinestretch=1,
	bgcolor=listingbg,
	fontsize=\scriptsize,
	linenos,
	breaklines
	}


\usepackage[american]{babel}
\usepackage{csquotes}
\usepackage[style=apa, backend = biber]{biblatex}
\renewcommand*{\bibfont}{\tiny}


\usepackage{tikz}
\usetikzlibrary{shapes,arrows,matrix}
\usepackage{multicol}

\usepackage{subcaption}

\usepackage{booktabs}
\usepackage{graphicx}



\makeatletter
\setbeamertemplate{headline}{%
	\begin{beamercolorbox}[colsep=1.5pt]{upper separation line head}
	\end{beamercolorbox}
	\begin{beamercolorbox}{section in head/foot}
		\vskip2pt\insertnavigation{\paperwidth}\vskip2pt
	\end{beamercolorbox}%
	\begin{beamercolorbox}[colsep=1.5pt]{lower separation line head}
	\end{beamercolorbox}
}
\makeatother





\setbeamercolor{section in head/foot}{fg=normal text.bg, bg=structure.fg}


\newcommand{\instruction}[1]{\emph{\textcolor{gray}{[#1]}}}



\newcommand{\question}[1]{
	\begin{frame}[plain]
	\begin{columns}
		\column{.3\textwidth}
		\makebox[\columnwidth]{
			\includegraphics[width=\columnwidth,height=\paperheight,keepaspectratio]{mannetje.png}}
		\column{.7\textwidth}
		\large
		\textcolor{orange}{\textbf{\emph{#1}}}
	\end{columns}
\end{frame}}


\tikzstyle{block} = [rectangle, draw, fill=blue!20, 
text width=5em, text centered, rounded corners, minimum height=4em]
\tikzstyle{line} = [draw]
\tikzstyle{pijltje} = [draw, -latex']
\tikzstyle{cloud} = [draw, ellipse,fill=red!20, node distance=3cm,
minimum height=2em, text width=4em, text centered,]


\setbeamercovered{transparent}

\addbibresource{../../resources/literature.bib}
\graphicspath{{../../resources/img/}}


\begin{document}

\title[Big Data and Automated Content Analysis]{\textbf{Big Data and Automated Content Analysis (6EC)} 
\\Week 5: »Unsupervised machine learning«
\\Monday}
\author[Anne Kroon]{Anne Kroon\\ \footnotesize{a.c.kroon@uva.nl, @annekroon \\}}
\date{October 2, 2023}
\institute[UvA CW]{UvA RM Communication Science}


\begin{frame}{}
	\titlepage
\end{frame}

\begin{frame}{Today}
	\tableofcontents
\end{frame}

\question{Everything clear from previous weeks? Questions?}

\begin{frame}[standout]
This week, we will get a general overview of working with textual data. Due to a lack of time, I will introduce you to some of the basic concepts, point you to resources, and give you a practical, hands-on introduction. 
\end{frame}

\begin{frame}{\cite{Boumans2016}: Types of Automated Content Analysis}
	\makebox[\columnwidth]{	\includegraphics[width=\columnwidth,height=.8\paperheight,keepaspectratio]{boumanstrilling2016}}
\end{frame}

\begin{frame}{Bottom-up vs. top-down}
\begin{block}{Bottom-up}
	\begin{itemize}
		\item Count most frequently occurring words 
		\item Maybe better: Count combinations of words $\Rightarrow$ Which words co-occur together?
	\end{itemize}
	We \emph{don't} specify what to look for in advance	
\end{block}

\onslide<2>{
	\begin{block}{Top-down}
		\begin{itemize}
			\item Count frequencies of pre-defined words
			\item Maybe better: patterns instead of words
		\end{itemize}
		We \emph{do} specify what to look for in advance	
	\end{block}
}

\end{frame}

\begin{frame}[fragile]{A simple bottom-up approach}

\begin{minted}[%
	breaklines,
	linenos,
	fontsize=\scriptsize,
	frame=single,
	xleftmargin=0pt,]
	{python}
from collections import Counter
texts = ["Communication in the Digital Society is a very very complex phenomenon", "I like to study it"]
bottom_up = []
for t in texts:
    bottom_up.append(Counter(t.lower().split()).most_common(3))
    print(bottom_up)
\end{minted}
\pause
This results in:
\begin{minted}[fontsize=\scriptsize]{python}
[('very', 2), ('Communication', 1), ('in', 1)]
[('I', 1), ('like', 1), ('to', 1)]
\end{minted}
\textcolor{red}{\footnotesize{\emph{Please note that you can also write this like:}}}
\pause
\begin{minted}[%
	breaklines,
	linenos,
	fontsize=\scriptsize,
	frame=single,
	xleftmargin=0pt,]
	{python}
bottom_up = [Counter(t.split()).most_common(3) for t  in texts]
\end{minted}
\begin{itemize}
	\tiny{
		\item This is \emph{exactly} the same, just shorter (and faster). 
		\item You do \emph{not} have to use list comprehensions, but it helps if you can read them. }
\end{itemize}

\end{frame}

\begin{frame}[fragile]{A simple top-down approach}
\begin{minted}[%
	breaklines,
	linenos,
	fontsize=\scriptsize,
	frame=single,
	xleftmargin=0pt,]
	{python}
texts = ["Communication in the Digital Society is a very very complex phenomenon", "I like to study it"]
features = ["communication", "digital", "study"]
for t in texts:
    print(f"\nAnalyzing '{t}':")
    for f in features:
        print(f"{f} occurs {t.lower().count(f)} times")
\end{minted}
\pause
\begin{minted}[fontsize=\tiny]{python}
Analyzing 'Communication in the Digital Society is a very very complex phenomenon':
communication occurs 1 times
digital occurs 1 times
study occurs 0 times
	
Analyzing 'I like to study it':
communication occurs 0 times
digital occurs 0 times
study occurs 1 times	
\end{minted}
\end{frame}

\question{When would you use which approach?}

\begin{frame}{Some considerations}
\begin{itemize}[<+->]
	\item Both can have a place in your workflow (e.g., bottom-up as first exploratory step)
	\item You have a clear theoretical expectation? Bottom-up makes little sense.
	\item But in any case: you need to transform your text into something ``countable''.
\end{itemize}
\end{frame}

\section{Unsupervised Machine Learning for Text Classification}


\begin{frame}{Supervised vs Unsupervised}
	\begin{columns}[t]
		\column{.5\textwidth}
		
		\begin{block}<1-4>{Supervised machine learning}
			You have a dataset with both predictor and outcome (independent and dependent variables; features and labels) --- a \emph{labeled} dataset.
			\onslide<2>{
				\footnotesize{Think of regression: You measured \texttt{x1}, \texttt{x2}, \texttt{x3} and you want to predict \texttt{y}, which you also measured}}
		\end{block}
		
		\column{.5\textwidth}
		
		\begin{block}<3->{Unsupervised machine learning}
			You have no labels. \onslide<4>{(\footnotesize{You did not measure \texttt{y})}}\\
			\onslide<5>{\textbf{You might already know \emph{some} techniques to figure out whether \texttt{x1}, \texttt{x2},\ldots \texttt{x\_i} co-occur} \begin{itemize}
					\item Principal Component Analysis (PCA) and Singular Value Decomposition (SVD)
					\item Cluster analysis
					\item Topic modelling (Non-negative matrix factorization and Latent Dirichlet Allocation)
					\item \ldots
			\end{itemize}}
		\end{block}
		
	\end{columns}
	
\end{frame}


\section[Unsupervised ML]{Unsupervised machine learning for text}


\question{Recap: Can you explain the difference between supervised and unsupervised machine learning?}



\begin{frame}{\cite{Boumans2016}: Types of Automated Content Analysis}
	\makebox[\columnwidth]{	\includegraphics[width=\columnwidth,height=.8\paperheight,keepaspectratio]{boumanstrilling2016}}
\end{frame}



\begin{frame}[standout]
Our goal is to identify topics in texts, but we \emph{do not know the topics in advance}.\\ \textcolor{orange}{If you do have theoretical expectations, use classic SML (or fine-tune a Transformer, maybe with few-/zero-shot learning.) instead.}

\end{frame}




\subsection{A historical overview: PCA, k-means, LDA}


\begin{frame}{Defining the problem}
Remember our earlier distinction:
\begin{enumerate}
\item Finding similar variables (dimension reduction)
\item Finding similar cases (clustering)
\end{enumerate}

\pause

Are we more interested in which features ``belong together'' or which cases ``belong together''? 

\pause

\textcolor{orange}{Conceptually, we want to know \emph{both} which features (words) belong to each other (=form a topic),  \emph{and} which cases (documents) contain the same topics.}

\end{frame}

\begin{frame}[fragile]{Defining the problem}
We assume a BOW approach like this (as produced by scikit-learn vectorizer):
    
\emph{Document-term matrix}
\begin{lstlisting}
      w1,w2,w3,w4,w5,w6 ...
text1, 2, 0, 0, 1, 2, 3 ...
text2, 0, 0, 1, 2, 3, 4 ...
text3, 9, 0, 1, 1, 0, 0 ...
...
\end{lstlisting}
    {\small{raw counts or tf$\cdot$idf scores}}

 \end{frame}
  

\begin{frame}{Defining the problem}
  We could then go via two routes:
  \begin{enumerate}[<+->]
    \item We run a PCA/SVD to see which features (words) load on the same component; and \emph{then} look at the component scores per document
    \item We run a $k$-means cluster analysis to see which texts are similar; and \emph{then} look at the most common words per cluster
  \end{enumerate}
\end{frame}

\begin{frame}{Some considerations}
  \begin{block}{If we do PCA/SVD\ldots}
  \begin{itemize}
  \item Components are ordered (first explains most variance) $\Rightarrow$ We assume that some topics are more important than others
  \item Components do \emph{not} necessarily carry a meaningful interpretation $\Rightarrow$ But maybe OK in practice?
  \item We assume that a word belongs to one (not multiple) topics
  \item We assume that a document has a score for each topic
\end{itemize}
  \end{block}
\end{frame}


\begin{frame}{Some considerations}
  \begin{block}{If we do cluster analysis\ldots}
  \begin{itemize}
  \item We assume that (in the case of $k$-means) that topics (are roughly) simiarly sized
  \item We assume that a \emph{document} belongs to one (not multiple) topics
  \item We assume that a word \emph{can} belong to multiple topics.
\end{itemize}
  \end{block}
\end{frame}



\begin{frame}{Both approaches have been used}
  \begin{itemize}
  \item Both have different assumptions, implications, and constraints
  \item Both easy to do with scikit-learn
  \item Both used in research (for instance, PCA by   \textcite{Leydesdorff2017} or \textcite{Greussing2017}; or $k$-means cluster analysis by \textcite{burscher2016})
  \item Typically, PCA groups features, cluster analysis groups texts --  (but you can then use the component scrores to describe the texts, and the cluster centroids to describe the features)
  \item \textcolor{orange}{Still ocasionally used, but in general considered outdated}
  \end{itemize}
  
\end{frame}




\begin{frame}[standout]
You find some slides with code examples in the appendix.
\end{frame}









%\subsection[LDA]{Towards Latent Dirichlet Allocation (LDA)}



\begin{frame}{Beyond PCA and $k$-means}

  PCA was invented in 1901 (!), and $k$-means is around since the 1950s/1960s.

  There surely must be something newer!

  \pause

  There is: Latent Dirichlet Allocation (LDA) \parencite{Blei2003}.

\end{frame}


\begin{frame}[fragile]{LDA solves some problems}
Actually, we have \emph{two} things we want to model:

\begin{enumerate}
	\item Which topics can we extract from the corpus?
	\item How present is each of these topics in each text in the corpus?
\end{enumerate}
$\Rightarrow$ LDA does both simultaneously!

It also does not suffer from a few problems:
\begin{itemize}
\item  \textcolor{red}{does the goal of PCA, to find a solution in which one word loads on \emph{one} component match real life, where a word can belong to several topics or frames?}
\item \textcolor{red}{does the goal of cluster analysis, assigning each document to \emph{one} cluster, match real life?}
\end{itemize}

\end{frame}


\begin{frame}{LDA solves some problems}
LDA is a model that 
\begin{enumerate}[<+->]
	\item estimates \emph{simultaneously} (a) which topics we find in the whole corpus, and (b) which of these topics are present in which document; while at the same time
	\item allowing (a) words to be part of multiple topics, and (b) multiple topics to be present in one document; and
	\item being able to make connections between words ``even if they never actually occured in a document together'' \parencite[p.~96]{Maier2018a}
\end{enumerate}
\pause
\textcolor{orange}{Let that last point sink in for a second!}
\end{frame}




\begin{frame}{LDA, what's that?}
  \begin{block}{No mathematical details here, but the general idea}
    \begin{itemize}
    \item There are $k$ topics, $T_1$\ldots$T_k$
    \item Each document $D_i$ consists of a mixture of these topics, e.g.$80\% T_1, 15\% T_2, 0\% T_3, \ldots 5\% T_k $
    \item On the next level, each topic consists of a specific probability distribution of words
    \item Thus, based on the frequencies of words in $D_i$, one can infer its distribution of topics
    \item Note that LDA (like PCA) is a Bag-of-Words (BOW) approach
    \end{itemize}
  \end{block}
	
\end{frame}




\begin{frame}[fragile]{Doing a LDA in Python}
You can use gensim \cite{Rehurek2010} for this.

Let us assume you have a list of lists of words (!) called \texttt{texts}:

\begin{lstlisting}
articles=['The tax deficit is higher than expected. This said xxx ...', 'Germany won the World Cup. After a']
texts=[[token for token in re.split(r"\W", art) if len(token)>0] for art in articles]
\end{lstlisting}
which looks like this:
\begin{lstlisting}
[['The', 'tax', 'deficit', 'is', 'higher', 'than', 'expected', 'This', 'said', 'xxx'], ['Germany', 'won', 'the', 'World', 'Cup', 'After', 'a']]
\end{lstlisting}
(note that we of course could use a better tokenizer!)
\end{frame}




\begin{frame}[plain,fragile]
\begin{lstlisting}
from gensim import corpora, models
import pandas as pd

NTOPICS = 100
LDAOUTPUTFILE="topicscores.tsv"

# Create a BOW represenation of the texts
id2word = corpora.Dictionary(texts)
mm =[id2word.doc2bow(text) for text in texts]

# Train the LDA models.
mylda = models.ldamodel.LdaModel(corpus=mm, id2word=id2word, num_topics=NTOPICS, alpha="auto")

# Print the topics.
for top in mylda.print_topics(num_topics=NTOPICS, num_words=5):
  print ("\n",top)

# the topic scores per document
topics = pd.DataFrame([dict(mylda.get_document_topics(doc, minimum_probability=0.0)) for doc in mm])

\end{lstlisting}
\end{frame}


\begin{frame}[fragile]{Output: Topics (below) \& topic scores (next slide)}
\begin{lstlisting}
0.069*fusie + 0.058*brussel + 0.045*europesecommissie + 0.036*europese + 0.023*overname
0.109*bank + 0.066*britse + 0.041*regering + 0.035*financien + 0.033*minister
0.114*nederlandse + 0.106*nederland + 0.070*bedrijven + 0.042*rusland + 0.038*russische
0.093*nederlandsespoorwegen + 0.074*den + 0.036*jaar + 0.029*onderzoek + 0.027*raad
0.099*banen + 0.045*jaar + 0.045*productie + 0.036*ton + 0.029*aantal
0.041*grote + 0.038*bedrijven + 0.027*ondernemers + 0.023*goed + 0.015*jaar
0.108*werknemers + 0.037*jongeren + 0.035*werkgevers + 0.029*jaar + 0.025*werk
0.171*bank + 0.122* + 0.041*klanten + 0.035*verzekeraar + 0.028*euro
0.162*banken + 0.055*bank + 0.039*centrale + 0.027*leningen + 0.024*financiele
0.052*post + 0.042*media + 0.038*nieuwe + 0.034*netwerk + 0.025*personeel
...
\end{lstlisting}
\end{frame}


\begin{frame}[plain]
\makebox[\linewidth]{
	\includegraphics[width=\paperwidth,height=\paperheight,keepaspectratio]{topicscores}}
\end{frame}




\begin{frame}[fragile]{Visualization with pyldavis}
\begin{lstlisting}
import pyLDAvis
import pyLDAvis.gensim_models as gensimvis
# first estiate gensim model, then:
vis_data = gensimvis.prepare(mylda,mm,id2word)
pyLDAvis.display(vis_data)
\end{lstlisting}
\makebox[\linewidth]{
	\includegraphics[width=\paperwidth,height=.5\paperheight,keepaspectratio]{pyldavis}}
\end{frame}

\begin{frame}{Visualization with pyldavis}
Short note about the $\lambda$ setting:

It influences the ordering of the words in pyldavis.

\begin{quote}
``For $\lambda = 1$, the ordering of the top words is equal to the ordering of the standard conditional word probabilities. For $\lambda$ close to zero, the most specific words of the topic will lead the list of top words. In their case study, Sievert and Shirley (2014, p. 67) found the best interpretability of topics using a  $\lambda$-value close to .6, which we adopted for our own case'' \parencite[p.~107]{Maier2018a}
\end{quote}

\end{frame}




\begin{frame}{Choosing the best (or a good) topic model}
\begin{itemize}
\item There is no single best solution (e.g., do you want more coarse of fine-grained topics?)
\item Non-deterministic
\item Very sensitive to preprocessing choices
\item Interplay of both metrics and (qualitative) interpretability 
\end{itemize}

See for more elaborate guidance:

\tiny{Maier, D., Waldherr, A., Miltner, P., Wiedemann, G., Niekler, A., Keinert, A., \ldots Adam, S. (2018). Applying LDA Topic Modeling in Communication Research: Toward a Valid and Reliable Methodology. \textit{Communication Methods and Measures, 12}(2--3), 93--118. doi:10.1080/19312458.2018.1430754}

\end{frame}



\begin{frame}{Evaluation metrics (closer to zero is better)}
\begin{block}{perplexity}
A goodness-of-fit measure, answering the question: If we do a train-test split, how well does the trained model fit the test data?
\end{block}

\pause 
\begin{block}{coherence}
\begin{itemize}
\item mean coherence of the whole model: attempts to quantify the interpretability
\item coherence per topic: allows to get topics that are most likely to be coherently interpreted (\texttt{.top\_topics()})
\end{itemize}
\end{block}

\end{frame}


\begin{frame}{So, how do we do this?}
\begin{itemize}[<+->]
	\item Basically, similar to the idea behind our grid search from two weeks ago: estimate multiple models, store the metrics for each model, and then compare them (numerically, or by plotting)
	\item Idea: We select some candidate models, and then look whether they can be interpreted.
	\item But what can we tune?
\end{itemize}
\end{frame}


\begin{frame}{Choosing $k$: How many topics do we want?}
\begin{itemize}
	\item Typical values: $10<k<200$
	\item Too low: losing nuance, so broad it becomes meaningless
	\item Too high: picks up tiny pecularities instead of finding general patterns
	\item There is no inherent ordering of topics (unlike PCA!)
	\item We can throw away or merge topics later, so if out of $k=50$ topics 5 are not interpretable and a couple of others overlap, it still may be a good model
\end{itemize}
\end{frame}


\begin{frame}[fragile]{Choosing $\alpha$: how sparse should the document-topic distribution $\theta$ be?}
\begin{itemize}
	\item The higher $\alpha$, the more topics per document 
	\item Default: $1/k$
	\item But: We can explicitly change it, or -- really cool -- even learn $\alpha$ from the data (\texttt{alpha = "auto"})
\end{itemize}

\pause 

Takeaway: It takes longer, but you probably want to learn alpha from the data, using multiple passes:

\begin{lstlisting}
mylda LdaModel(corpus=tfidfcorpus[ldacorpus], id2word=id2word, num_topics=50, alpha='auto', passes=10)
\end{lstlisting}


\end{frame}


\begin{frame}{Choosing $\eta$: how sparse should the topic-word distribution $\lambda$ be?}
  \begin{itemize}
  \item Can be used to boost specific words
  \item Can also be learned from the data 
  \end{itemize}

\pause
Takeaway: Even though you can do \texttt{eta="auto"}, this usually does not help you much.

\end{frame}




\begin{frame}{Using topic models}

You got your model -- what now?

\begin{enumerate}
\item Assign topic scores to documents
\item Label topics
\item Merge topics, throw away boilerplate topics and similar (manually, or aided by cluster analysis)
\item Compare topics between, e.g., outlets
\item or do some time-series analysis.
\end{enumerate}


Example: \cite{Tsur2015}

\end{frame}






\subsection{Should one still use LDA?}


\begin{frame}{The popularity of LDA}
In the last decade, LDA has become \emph{extremely} popular in the social sciences due to
  \begin{itemize}
  \item easy-to-use R and Python packages
  \item its promise to not require (a) manual (qual or quant) analysis; (b) annotations for SML; (c) creation of dictionaries etc.
  \item a bit of a ``cool new technique'' image
  \end{itemize}
\end{frame}


\begin{frame}{The popularity of LDA}
  \textbf{But there is no silver bullet!}

  Unfortunately,
  \begin{itemize}
  \item validating topic models is hard -- and many (most) studies don't do it (well);
  \item there are so many choices and parameters, in combination with no simple and definite evaluation metric, that it is very hard to justify why a particular model is chosen;
  \item experience shows that it often ``doesn't work'' $\Rightarrow$ it's quite normal to have many uninterpretable or ambigous topics;
  \item The smaller the dataset, the less likely it is to work
  \item LDA tends to also pick up pecularities that don't matter and outliers
  \end{itemize}
\end{frame}


\begin{frame}{Solutions?}
There are some extensions on classical LDA, in particular:
\begin{itemize}
\item Author-topic models
\item Structural topic models (STM) \parencite{Roberts2014}
\item Dynamic topic models \parencite{Blei2006}
\end{itemize}

These allow covariates (e.g., add info on who wrote a text) to improve the model, or allow to account for the changing use of words and topics over time.

Also, there are techniques for validation available (e.g., topic intrusion and/or word intrusion tasks).
\end{frame}


\begin{frame}{Solutions?}
  But some we can't solve everything.

  \begin{itemize}
  \item It's still BOW.
  \item We cannot incorporate any language knowledge from larger, pre-trained datasets (e.g., via embeddings)
  \end{itemize}

$\Rightarrow$ If we think of the performance leap that we observe with Transformers in other areas, we have all reason to assume that we can do better.
\end{frame}



\subsection{State-of-the-art approaches to topic modelling}

\begin{frame}[standout]
Let's bring in embeddings and Transformers!
\end{frame}


\begin{frame}{Using embeddings and transformers for topic modelling}
  For example:
  \begin{itemize}[<+->]
  \item top2vec \parencite{angelov2020top2vec}, which embeds \emph{topic vectors} in the same space as document vectors and word vectors
  \item Contextualized Topic models \parencite{bianchi-etal-2021-pre,bianchi-etal-2021-cross}, with a lot of code examples at \url{https://contextualized-topic-models.readthedocs.io/en/latest/introduction.html}
  \item \ldots
    \end{itemize}
 
\end{frame}



\begin{frame}{BERTopic \parencite{Grootendorst2022}}
%Let's discuss one specific approach: BERTtopic. 

%\begin{block}{A high-level view}
``In this paper, we introduce BERTopic, a topic model that leverages clustering techniques and a class-based variation of TF-IDF to generate coherent topic representations. More specifically, we first create document embeddings using a pretrained language model to obtain document-level information. Second, we first reduce the dimensionality of document embeddings before creating semantically similar clusters of documents that each represent a distinct topic. Third, to overcome the centroid-based perspective, we develop a classbased version of TF-IDF to extract the topic representation from each topic. These three independent steps allow for a flexible topic model that can be used in a variety of use-cases, such as dynamic topic modeling.''
%\end{block}

\tiny{(for details, read the paper)}

\end{frame}




\begin{frame}{Much more coherent topics than LDA!}
  \makebox[\columnwidth]{	\includegraphics[width=\columnwidth,height=.8\paperheight,keepaspectratio]{grootendorst2022}}
  (And no need to set $k$! And there is a dedicated ``outlier topic'' called $-1$!)
\end{frame}


\begin{frame}{Are there downsides?}

  Of course!

  \begin{itemize}
  \item By definiton, much more ``black-box''-y than BOW approaches
  \item Risk of biases introduced by LLM
  \item Much more resource-hungry (you probably want to do this with a GPU (e.g., on CoLab)
  \end{itemize}
\end{frame}


\begin{frame}[standout]
To conclude: PCA, $k$-means, LDA are interesting starting points -- but if I were to start an unsupervised topic analysis model now, I'd go for BERTopic.

\end{frame}






\section[Appendix]{Appendix: Code examples}


\begin{frame}[fragile,plain]{}
  \begin{minted}{python}
from sklearn import datasets
from sklearn.decomposition import TruncatedSVD
from sklearn.feature_extraction.text import TfidfVectorizer
from sklearn.pipeline import make_pipeline

texts = datasets.fetch_20newsgroups(data_home='rec.autos', remove=('headers', 'footers', 'quotes'), subset='train')['data']

myvec = TfidfVectorizer(max_df=.5, min_df=5, token_pattern='(?u)\\b[a-zA-Z][a-zA-Z]+\\b')
mysvd = TruncatedSVD(n_components=3)
mypipe = make_pipeline(myvec, mysvd)
r = mypipe.fit_transform(texts)
\end{minted}
\end{frame}





\begin{frame}[fragile]{Plotting the texts according to their component scores}
\begin{lstlisting}
import matplotlib.pyplot as plt

fig = plt.figure(figsize=(10,10))
ax = fig.add_subplot(projection='3d')
ax.scatter([e[0] for e in r], [e[1] for e in r],  [e[2] for e in r],alpha=.2)
\end{lstlisting}

\makebox[\linewidth]{\includegraphics[width=\paperwidth,height=.6\paperheight,keepaspectratio]{svd3d}}

\end{frame}



\begin{frame}[fragile]{Using the scores}
\begin{lstlisting}
import pandas as pd
textscores= pd.DataFrame(r)
featurescores = pd.DataFrame(mysvd.components_.T, index = myvec.get_feature_names())
\end{lstlisting}

\makebox[\linewidth]{\includegraphics[width=\paperwidth,height=.6\paperheight,keepaspectratio]{svdscores}}

\end{frame}



\begin{frame}{Grouping features vs grouping cases}
  We have a  corpus of a many texts.
  
\begin{itemize}
\item We used SVD to figure out relationships between features
\item We could now look at the most important features per component (``topic'', ``frame''?) by sorting \texttt{featurescores}
\item We could see which texts are most representative for each ``topic'' or ``frame'' by sorting \texttt{textscores}
\end{itemize}
\pause

$\Rightarrow$ Alternative: Choose the opposite approach and first find out which cases are most similar, \textit{then} describe what features characterize each group of cases


\end{frame}





\begin{frame}[plain,fragile]
\begin{minted}{python}
from sklearn.cluster import KMeans

k = 5

mykm = KMeans(n_clusters=k, init='k-means++', max_iter=100, n_init=1)
myvec = TfidfVectorizer(max_df=.5, min_df=5, token_pattern='(?u)\\b[a-zA-Z][a-zA-Z]+\\b')
mypipe = make_pipeline(myvec, mykm)

predictions = mypipe.fit_transform(texts)
# potentially: textscores = pd.Dataframe(predictions)
\end{minted}

Of course, you need to determine the appropriate $k$\ldots (see earlier lecture)



\end{frame}


\begin{frame}[fragile,plain]{Let's get the terms closest to the centroids}
\begin{lstlisting}
order_centroids = mykm.cluster_centers_.argsort()[:, ::-1]
terms = myvec.get_feature_names()

print("Top terms per cluster:")

for i in range(k):
    print("Cluster {}: ".format(i), end='')
    for ind in order_centroids[i, :10]:
        print("{} ".format(terms[ind]), end='')
    print()

\end{lstlisting}
\pause
returns something like:

\begin{lstlisting}
Top terms per cluster:
Cluster 0: windows file dos window with on you this have files 
Cluster 1: you on this was with are have be not they 
Cluster 2: thanks any me have anyone or please if on this 
Cluster 3: he was his him not as this but on god 
Cluster 4: you are be not they as this have if on 
\end{lstlisting}
(of course, we could do sth similar with pandas as well)
\end{frame}








\question{Any questions?}

\section{Next steps}

\begin{frame}[standout]
Take-home exam: you have time until Thursday 5th, end of the day

An example notebook with code for running LDA models can be found here:
\large{\url{https://github.com/uvacw/teaching-bdaca/blob/main/6ec-course/week06/exercises/}}
\end{frame}



\begin{frame}[allowframebreaks,plain]
	\printbibliography
\end{frame}



\end{document}
