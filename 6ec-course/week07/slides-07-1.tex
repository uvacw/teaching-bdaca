\documentclass[compress]{beamer}
% !TeX document-id = {f19fb972-db1f-447e-9d78-531139c30778}
% !BIB program = biber

%\documentclass[handout]{beamer}
%\documentclass[compress]{beamer}
\usepackage[T1]{fontenc}
\usetheme[block=fill,subsectionpage=progressbar,sectionpage=progressbar]{metropolis} 
\usepackage{graphicx}

\usepackage{wasysym}
\usepackage{etoolbox}
\usepackage[utf8]{inputenc}

\usepackage{pifont}

\usepackage{threeparttable}
\usepackage{subcaption}

\usepackage{tikz-qtree}
\usepackage{neuralnetwork}

\setbeamercovered{still covered={\opaqueness<1->{5}},again covered={\opaqueness<1->{100}}}


\usepackage{listings}

\lstset{
	basicstyle=\scriptsize\ttfamily,
	columns=flexible,
	breaklines=true,
	numbers=left,
	%stepsize=1,
	numberstyle=\tiny,
	backgroundcolor=\color[rgb]{0.85,0.90,1}
}



\lstnewenvironment{lstlistingoutput}{\lstset{basicstyle=\footnotesize\ttfamily,
		columns=flexible,
		breaklines=true,
		numbers=left,
		%stepsize=1,
		numberstyle=\tiny,
		backgroundcolor=\color[rgb]{.7,.7,.7}}}{}


\lstnewenvironment{lstlistingoutputtiny}{\lstset{basicstyle=\tiny\ttfamily,
		columns=flexible,
		breaklines=true,
		numbers=left,
		%stepsize=1,
		numberstyle=\tiny,
		backgroundcolor=\color[rgb]{.7,.7,.7}}}{}


% color-coded listings; replace those above 
\usepackage{xcolor}
\usepackage{minted}
\definecolor{listingbg}{rgb}{0.87,0.93,1}
\setminted[python]{
	frame=none,
	framesep=1mm,
	baselinestretch=1,
	bgcolor=listingbg,
	fontsize=\scriptsize,
	linenos,
	breaklines
	}


\usepackage[american]{babel}
\usepackage{csquotes}
\usepackage[style=apa, backend = biber]{biblatex}
\renewcommand*{\bibfont}{\tiny}


\usepackage{tikz}
\usetikzlibrary{shapes,arrows,matrix}
\usepackage{multicol}

\usepackage{subcaption}

\usepackage{booktabs}
\usepackage{graphicx}



\makeatletter
\setbeamertemplate{headline}{%
	\begin{beamercolorbox}[colsep=1.5pt]{upper separation line head}
	\end{beamercolorbox}
	\begin{beamercolorbox}{section in head/foot}
		\vskip2pt\insertnavigation{\paperwidth}\vskip2pt
	\end{beamercolorbox}%
	\begin{beamercolorbox}[colsep=1.5pt]{lower separation line head}
	\end{beamercolorbox}
}
\makeatother





\setbeamercolor{section in head/foot}{fg=normal text.bg, bg=structure.fg}


\newcommand{\instruction}[1]{\emph{\textcolor{gray}{[#1]}}}



\newcommand{\question}[1]{
	\begin{frame}[plain]
	\begin{columns}
		\column{.3\textwidth}
		\makebox[\columnwidth]{
			\includegraphics[width=\columnwidth,height=\paperheight,keepaspectratio]{mannetje.png}}
		\column{.7\textwidth}
		\large
		\textcolor{orange}{\textbf{\emph{#1}}}
	\end{columns}
\end{frame}}


\tikzstyle{block} = [rectangle, draw, fill=blue!20, 
text width=5em, text centered, rounded corners, minimum height=4em]
\tikzstyle{line} = [draw]
\tikzstyle{pijltje} = [draw, -latex']
\tikzstyle{cloud} = [draw, ellipse,fill=red!20, node distance=3cm,
minimum height=2em, text width=4em, text centered,]


\setbeamercovered{transparent}

\addbibresource{../../resources/literature.bib}
\graphicspath{{../../resources/img/}}


\begin{document}

\title[Big Data and Automated Content Analysis]{\textbf{Big Data and Automated Content Analysis (6EC)} 
\\Week 7: »Supervised Approaches to Text Analysis«
\\Monday}
\author[Anne Kroon]{Anne Kroon\\ \footnotesize{a.c.kroon@uva.nl, @annekroon \\}}
\date{May 15, 2022}
\institute[UvA CW]{UvA RM Communication Science}


\begin{frame}{}
	\titlepage
\end{frame}

\begin{frame}{Today}
	\tableofcontents
\end{frame}

\question{Everything clear from last week?}


\begin{frame}[standout]
This week, we will get a general overview of working with textual data. Due to a lack of time, I will introduce you to some of the basic concepts, point you to resources, and give you a practical, hands-on introduction. 
\end{frame}


\section{Supervised Machine Learning for Text Classification}

\subsection{You have done it before!}
\begin{frame}{You have done it before!}
	\begin{block}{Regression}<2->
		\begin{enumerate}
			\item<3-> Based on your data, you estimate some regression equation 	$y_i = \alpha + \beta_1 x_{i1} + \cdots + \beta_p x_{ip} + \varepsilon_i$
			\item<4-> Even if you have some \emph{new unseen data}, you can estimate your expected outcome $\hat{y}$!
			\item<5-> Example: You estimated a regression equation where $y$ is newspaper reading in days/week: $y = -.8 + .4 \times man + .08 \times age$
			\item<6-> You could now calculate $\hat{y}$ for a man of 20 years and a woman of 40 years -- \emph{even if no such person exists in your dataset}: \\
			$\hat{y}_{man20} = -.8 + .4 \times 1 + .08 \times 20 = 1.2$ \\
			$\hat{y}_{woman40} = -.8 + .4 \times 0 + .08 \times 40 = 2.4$
		\end{enumerate}
	\end{block}	
	
\end{frame}

\begin{frame}{}
	\huge{This is\\ Supervised Machine Learning!}
\end{frame}

\begin{frame}{\ldots but\ldots}
	\begin{itemize}
		\item<1-> We will only use \emph{half} {\tiny{(or another fraction)}} of our data to estimate the model, so that we can use the other half to check if our predictions match the manual coding (``labeled data'',``annotated data'' in SML-lingo)
		\begin{itemize}
			\item<2->e.g., 2000 labeled cases, 1000 for training, 1000 for testing --- if successful, run on 100,000 unlabeled cases
		\end{itemize}
		\item<3-> We use many more independent variables (``features'')
		\item<4-> Typically, IVs are word frequencies (often weighted, e.g. tf$\times$idf) ($\Rightarrow$BOW-representation)
	\end{itemize}
\end{frame}


\subsection{From regression to classification}

\begin{frame}[standout]
	In the machine learning world, predicting some continous value is referred to as a \textcolor{orange}{regression} task. If we want to predict a binary or categorical variable, we call it a \textcolor{orange}{classification} task.
	
	(quite confusingly, even if we use a logistic regression for the latter)
\end{frame}


\begin{frame}{Classification tasks}
	For many computational approaches, we are actually not that interested in predicting a continous value. Typical questions include:
	\begin{itemize}
		\item Is this article about topic A, B, C, D, or E?
		\item Is this review positive or negative?
		\item Does this text contain frame F?
		\item Is this satire? 
		\item Is this misinformation?
		\item Given past behavior, can I predict the next click?
	\end{itemize}
\end{frame}



\begin{frame}[plain]
	\begin{columns}[]
		\column{.5\textwidth}
		
		{\tiny{http://commons.wikimedia.org/wiki/File:Precisionrecall.svg}}
		\makebox[\linewidth]{
			\includegraphics[width=\paperwidth,height=\paperheight,keepaspectratio]{../../resources/img/precisionrecall.png}}
		
		\column{.5\textwidth}
		\begin{block}{Some measures}
			\begin{itemize}
				\item Accuracy
				\item Recall
				\item Precision
				\item $\text{F1}=2\cdot \frac{\text{precision}\cdot \text{recall}}{\text{precision}+\text{recall}}$
				\item AUC (Area under curve) $[0,1]$, $0.5=$ random guessing
			\end{itemize}
		\end{block}
		
	\end{columns}
	
\end{frame}


\input{../../modules/machinelearning-text/sml1.tex}

\question{Any questions?}

\section{Next steps}

\begin{frame}[standout]
I prepared exercises to work on \emph{during} the Thursday meeting (alone or in teams):
\large{\url{https://github.com/uvacw/teaching-bdaca/blob/main/6ec-course/week07/exercises/}}
\end{frame}

\begin{frame}[standout]
Next monday: time for your individual questions about the final project. 
Sign up via Canvas!
\end{frame}



\begin{frame}[allowframebreaks,plain]
	\printbibliography
\end{frame}



\end{document}
