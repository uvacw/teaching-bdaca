\section{Basic string operations}

\begin{frame}{Working with strings}
\begin{enumerate}[<+->]
\item string methods that every string has (\texttt{"hello".upper()})
\item functions that take a string as input (\texttt{len("hello")})
\item pandas column string methods (\texttt{df["somecolumn"].str.upper()})
\item applying string methods or functions to a pandas column (\texttt{df["somecolumn"].apply(len)} or \texttt{df["somecolumn"].apply(lambda x: x.upper()})
\end{enumerate}

\pause
For today, we assume that our data are a list of strings -- adapt accordingly for pandas.
\end{frame}


\begin{frame}[fragile]{An example says more than 1000 words\ldots}
\begin{minted}{python}
# probably read from text file(s) instead, you learned that already...
data = [ "I <b>really</b> liked this movie! It was great.  ", "  What an awful movie", "Awesome!!!"]

data_stripped = [e.strip() for e in data]
data_lower = [e.lower() for e in data_stripped]
data_clean = [e.replace("<b>",").replace("</b>",") for e in data_lower]

# or, more efficient, in one single step:
data_clean2 = [e.strip().lower().replace("<b>","").replace("</b>","") for e in data]
\end{minted}
\end{frame}



\begin{frame}[fragile]{Two examples says even more:}
\begin{minted}{python}
from string import punctuation

# punctuation is just the string '!"#$%&\'()*+,-./:;<=>?@[\\]^_`{|}~'

text = "This is a test! Let's get rid (of) punct&"

# we make a list of each character in the text but only if it is not
# a punctuation sign. The, we join the elements of the list directly
# to each other without anything between it ("")
cleantext = "".join([c for c in text if c not in punctuation])
\end{minted}
\end{frame}


\begin{frame}[fragile]{Combine both}
\begin{minted}{python}
from string import punctuation

def strip_punctuation(text):
    return "".join([c for c in text if c not in punctuation])

data_clean3 = [strip_punctuation(e).strip().lower()\
   .replace("<b>","").replace("</b>","") for e in data]
    
\end{minted}
\end{frame}




\begin{frame}{The toolbox at a glance}
  \footnotesize
\begin{block}{Slicing}
\texttt{mystring[2:5]} to get the characters with indices 2,3,4
\end{block}

\begin{block}{String methods}
\begin{itemize}
	\item \texttt{.lower()} returns lowercased string
	\item \texttt{.strip()} returns string without whitespace at beginning and end
	\item \texttt{.find("bla")} returns index of position of substring ``bla'' or -1 if not found
	\item \texttt{.replace("a","b")} returns string with "a" replaced by "b"
	\item \texttt{.count("bla")} counts how often substring ``bla'' occurs
        \item \texttt{.isdigit()} true if only numbers
          
\end{itemize}
Use tab completion for more!
\end{block}
\end{frame}




\begin{frame}[fragile]{From test to large-scale analysis: General approach}
1. Take a single string and test your idea
\begin{lstlisting}
t = "This is a test test test."
print(t.count("test"))
\end{lstlisting}
2a. You'd assume it to return 3. If so, scale it up:
\begin{lstlisting}
results = []
for t in listwithallmytexts:
    r = t.count("test")
    print(f"{t} contains the substring {r} times")
    results.append(r)
\end{lstlisting}

2b. If you \emph{only} need to get the list of results, a list comprehension is more elegant:
\begin{lstlisting}
results = [t.count("test") for t in listwithallmytexts]
\end{lstlisting}


\end{frame}


\begin{frame}[fragile]{General approach}
\Large

\textcolor{red}{Test on a single string, then make a for loop or list comprehension!}

\pause

\normalsize

\begin{alertblock}{Own functions}
If it gets more complex, you can write your own function and then use it in the list comprehension:
\begin{lstlisting}
def mycleanup(t):
   # do sth with string t here, create new string t2
   return t2
  
results = [mycleanup(t) for t in allmytexts]
\end{lstlisting}
\end{alertblock}
\end{frame}


\begin{frame}[fragile]{Pandas string methods as alternative}
If you select column with strings from a pandas dataframe, pandas offers a collection of string methods (via \texttt{.str.}) that largely mirror standard Python string methods:

\begin{lstlisting}
df['newcoloumnwithresults'] = df['columnwithtext'].str.count("bla")
\end{lstlisting} 


\pause

\begin{alertblock}{To pandas or not to pandas for text?}
Partly a matter of taste. 

Not-too-large dataset with a lot of extra columns? Advanced statistical analysis planned? Sounds like pandas.

It's mainly a lot of text? Wanna do some machine learning later on anyway? It's large and (potentially) messy? Doesn't sound like pandas is a good idea.
\end{alertblock}

\end{frame}

