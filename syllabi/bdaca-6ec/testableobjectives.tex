
{\footnotesize{
3. Knowledge and Understanding: Have in-depth knowledge and a thorough understanding of advanced research designs and methods. 


3.1. Have in-depth knowledge and a thorough understanding of advanced research designs and methods, including their value and limitations.


3.2.	Have in-depth knowledge and a thorough understanding of advanced techniques for data analysis.

}}

\begin{enumerate}[A]
\item Students can explain \emph{fundamental research designs} and \emph{commonly employed methods} in existing research articles on Big Data and automated content analysis.
\item Students can on their own and in own words critically discuss the pros and cons of \emph{fundamental} research designs and methods employed in existing research articles on Big Data and automated content analysis; they can, based on this, give a critical evaluation of the methods and, where relevant, give advice to improve the study in question.
\item Students are able to identify some \emph{basic} techniques from the field of computer science and computer linguistics that are applicable to research in communication science; they can explain the principle of some traditional approaches to text analysis, namely simple rule-based techniques and basic methods of unsupervised and supervised machine learning. 
\end{enumerate}

{\footnotesize{
4.	Skills and abilities: Are able, independently and on their own, to set up, conduct, report and interpret advanced academic research.

4.1	Are able to formulate research questions and hypotheses for advanced empirical studies


4.2	Are able to develop a research plan, choose appropriate and suitable research designs and methods for advanced empirical studies, and justify the underlying choices. 


4.3	Are able to assess the validity and reliability of advanced empirical research, and to judge the scientific and professional value of findings from advanced empirical research.


4.4	Are able to apply advanced empirical research methods.

 }}

\begin{enumerate}[A]
\setcounter{enumi}{3}
\item Students can on their own formulate a research question and hypotheses for own empirical research in the domain of Big Data.
\item Students can on their own chose, execute and report on \emph{fundamental} research methods in the domain of Big Data and automatic content analysis.
\item Students know how to collect data with APIs or read in existing data files; they know how to analyze these data with fundamental automated techniques and to this end, they have basic knowledge of the programming language Python and know how to use fundamental Python-modules for communication science research.
\end{enumerate}


{\footnotesize{
6. Academic attitudes

6.1 	Regularly asses their own assumptions, strengths and weaknesses critically.


6.2	Accept that scientific knowledge is always 'work in progress' and that something
regarded as 'true' may be proven to be false, and vice-versa.


6.3 	Are keen to acquire new knowledge, skills and abilities. 


6.4 	Are willing to share and discuss arguments, results and conclusions, including submitting one's own work to peer review. 


6.5 	Are convinced that academic debates should not be conducted on the basis of rhetorical qualities but that arguments must be considered and conclusions drawn on the basis of empirical results and valid criticism.

 }}

\begin{enumerate}[A]
\setcounter{enumi}{6}
\item Students can critically discuss strong and weak points of their own research using fundamental techniques from the field of Big Data and Automated Content Analysis, and suggest improvements.
\item Students participate actively: reading the literature carefully and on time, completing assignments carefully and on time, active participation in discussions, and giving feedback on the work of fellow students give evidence of this.
\end{enumerate}

