The final grade of this course will be composed of the grade of one mid-term take home exam (30\%) and one individual project (70\%).

\subsection*{Mid-term take-home exam (30\%)}
In the mid-term take-home exam, students will show their understanding of the literature and prove they have gained new insights during the lecture/seminar meetings. They will be asked to critically assess various approaches to Big Data analysis and make own suggestions for research. Additionally, they need to (partly) write the code to accomplish this.

Grading criteria are communicated to the students together with the assignment, but in general are:
For literature-related tasks in the exam:

\begin{itemize}
	\item usage of specific examples from the literature;
	\item critique of different approaches;
	\item nameing of pro's, con's, potential pitfalls, and alternatives;
	\item giving practical advice and guidance.
\end{itemize}
For programming-related tasks in the exam:
\begin{itemize}
	\item correctness, efficiency, and style of the code
	\item correctness, completeness, and usefulness of analyses applied.
\end{itemize}
For conceptual and planning-related tasks:
\begin{itemize}
	\item feasibility
	\item level of specificity
	\item explanation and argumentation why a specific approach is chosen
	\item creativity.
\end{itemize}

\subsection*{Final individual project (70\%)}
The final individual project typically consists of the following elements, which all contribute to the final grade:
\begin{itemize}
\item introduction including references to relevant (course) literature, an overarching research question plus subquestions and/or hypotheses (1–2 pages);
\item an overview of the analytic strategy, referring to relevant methods learned in this course;
\item carefully collected and relevant dataset of non-trivial size; here, using APIs, or combining existing datasets. 
\item a set of scripts for collecting, preprocessing, and analyzing the data using fundamental techniques discussed in the course. The scripts should be well-documented and tailored to the specific needs of the own project;
\item output files;
\item a well-substantiated conclusion with an answer to the RQ and directions for future research.
\end{itemize}

Depending on the choosen topic, the student will have to apply some of the techniques covered in the course. The assignment needs to present an explorative description of the dataset, combined with some of the fundamental techniques discussed during the course. During the lab sesions, as well as individual consultation, students may discuss the scope of the projects with the lecturer, the requirements that the specific project suggested by the student needs to fulfill, and the extend to which the different methods that the student plans to use will contribute to the final grade.

\subsection*{Grading and 2\textsuperscript{nd} try}
Students have to get a pass (5.5 or higher) for both mid-term take-home exam and the individual project. If the grade of one of these is lower, an improved version can be handed in within one week after the grade is communicated to the student. If the improved version still is graded lower than 5.5, the course cannot be completed. Improved versions of the final individual project cannot be graded higher than 6.0. 
